%---
\section{Environmental, Safety and Health}
%\cmt{Lead Editor}{Federico Gabriele/Roberto Tartaglia}
%\cmt{Number of pages}{1-2}

%---
\subsection{Environmental, Safety and Health Plans}
%The Collaboration aims at establishing outstanding performance in matters of Health, Safety and Environment (HSE). The whole experimental collaboration is involved in the effort to achieve these targets. The collaboration will rely on skilled researchers, technicians and professionals with proven experience in the design, construction, commissioning and operation of complex plants. All aspect of the experiment, both technical and managerial, are handled in conformity to all applicable US, European and Italian HSE Regulations and Standards, as well as the national laboratory internal guidelines and procedures.

The Collaboration is developing an environment, safety, and health plan in cooperation with the host laboratory ESH Staff that is in accordance with ISO and OHSAS codes and conforms to all applicable US, European, and Italian HSE Regulations and Standards, as well as the national laboratory internal guidelines and procedures. For each sub-project, the collaboration will perform a PRA (Preliminary Risk Assessment), build a QRA (Quantitative Risk Assessment), and define a safety plan that is reviewed and revised as the project evolves. Development of the QRA will be carried out by experts from the relevant funding agencies, \GADMC\ member institutions, and external consultants. A similar scheme was successfully implemented for \DSf. The safety plan will cover the health and safety of all personnel and the potential environmental impacts of activities. 

%All the steps above is only a part of the whole process: all the site organization and safety has been already described in paragraph 11.


%The collaboration continues to develop and adopt its own policy statements and safety manage- ment planning documents, in accordance with the ISO and OHSAS code. In cooperation with the laboratory and site HSE Staff, Environmental and Safety handbooks, procedures and operating instructions are developed and carefully applied for relevant activities and for the management and operation of plants. A logbook will be continuously updated with the daily operations performed. Furthermore, information, education, and training of all Collaboration members are crucial aspects. Safety management and related information and procedures described are specific to the design, approval, installation, and commissioning of the various components, which make up the entire project. A safety risk assessment will be performed and fully analyzed before the construction and installation phases begin.
%The risk analysis methodologies used will be both qualitative and quantitative and the risk assessments will be carried out by the whole Collaboration and will be submitted to the laboratory safety personnel attention. Contributions to the safety risk assessment will be made by expert personnel belonging to the funding agencies, member Institutions of the collaboration, and also external consultants with proven expertise in the field of safety risk assessment and analysis. This tight cooperation ensured successful results in the past activity of DarkSide-50 and the aim is to follow the same path while striving to enhance, if possible, the performance in all matters.
%It is true that DarkSide-20k is a different order of magnitude bigger and more complicated than DarkSide-50. Moreover, DarkSide-20k also foresees the realization and construction of important facilities in places and sites far away from LNGS, such as the ARIA distillation column in Sardinia and the URANIA extraction plant in Colorado. For this reason, the Collaboration policy is to have the same behavior for each installation site. The common steps defined for each of the locations involved in the project are:
%? Performing a PRA (Preliminary Risk Assessment);
%26
%? Realizing a QRA (Quantitative Risk Assessment) before each installation; ? Defining a Safety Report;
%? Discussing and analyzing the Safety Report with the site owner for update; ? Improvements and integration from thing learned.
%The safety risk assessment and evaluation will take care of both health and safety of all the people involved, and of the possible impact on the environment that the installation may cause. The leading idea is to co-operate from the very beginning with the site hosting owner/institutions, in order to reduce the possible interferences with the local activities, and to study and determine all the preventive and protective measures needed to mitigate/reduce the risk and to plan well in advance the appropriate changes, upgrades and improvements of the auxiliary and infrastructure plants.
%The safety risk assessment will refer to the entire project and the evaluation will be performed in different areas:
%? Context (external) seismic condition,
%? Running of the process plants;
%? Physical and chemical agents (substances) adopted;
%? Working in confined areas and/or in underground condition; ? Explosion risk;
%? Handling of very heavy equipment.
%Once the initial report is finalized and improvements have been defined and agreed upon between all parties involved, a new risk assessment will be advised, in order to re-assess probability and impact and to check that the studied events and the risk associated with the identified events are in the range of acceptability in the defined matrix.




