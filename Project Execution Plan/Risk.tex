%---
\section{Risk and Opportunity Management}

%---
\subsection{Risk Management Plan}

The project is developing two risk assessments for the project, a preliminary risk assessment and a quantitative risk assessment.  The Preliminary Risk Assessment (PRA) uses qualitative and semi-quantitative methods and techniques, such as hazards identification (HazID), hazard and operability studies (HazOp), and failure modes and effects criticality analysis (FMEA/FMECA).  Using these techniques, risks will be identified and defined and a preliminary risk characterization will be developed, from which, a definition and adoption of preventive and protective measures will be developed.

The Quantitative Risk Analysis (QRA) builds on the PRA using quantitative methodologies and techniques, such as Fault Tree Analysis (FTA), Event Tree Analysis (ETA), and Consequence Modeling Analysis (CMA).  Potential adverse events will be identified and quantified with a likelihood of occurrence that provides a basis for evaluating the acceptability of risks and defining a risk matrix. In the risk matrix, each risk will be analyzed and evaluated to determine its probability and consequence, both economic and social. This will provide the management with a basis for understanding and defining corrective actions and strategies that ensure the successful completion of the project.

Our risk handling strategy can be summarized in a few sequential actions:

\begin{compactitem}
\item[\bf Avoid:] eliminate the event leading to the risk;
\item[\bf Transfer:] allocate the risk to a party more capable of dealing with it;
\item[\bf Control:] reduce (or eliminate) the likelihood, impact, or both of the risk to an acceptable level; 
\item[\bf Accept:] acknowledge that some risks are not avoidable and should be transferred and/or mitigated. 
\end{compactitem}

The opportunity handling strategy can be summarized as follows:

\begin{compactitem}
\item[\bf Enhance:] enhance the likelihood of the opportunity; 
\item[\bf Share:] develop teams/partnerships that will increase the opportunity's probability; 
\item[\bf Exploit:] increase (or facilitate) the impact, likelihood, or both of the opportunity;
\item[\bf Ignore:] ignore the opportunities with insufficient  return on investment; 
\item[\bf Place:] place on a watch-list and/or a check-list to monitor.
\end{compactitem}

Using a well defined methodology and tools to monitor, control, and audit risks and opportunities will increase our understanding of the project status and allow us to employ the most effective strategies to see the project to timely completion.

Since the project's inception, regular meetings have been organized to discuss the \WBS\ and the overall risk management plan in co-operation with the host laboratory, \LNGS.  Going forward, we will hold monthly meetings and employ internal and external consultants, who will review the project and provide advice on project risks and their concerns throughout the projects duration.