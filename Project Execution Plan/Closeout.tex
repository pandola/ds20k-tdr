%---
\section{Project Close-out}

%---
%---
\subsection{Transition to Operations Plan}

\textbf{\DSks:} We will use experiences from the operation of \DSf\ and \DSp\ to estimate the operational requirements for the \DSks\ detector. Standard operations of the \DSks\ detector will include full-time monitoring via software and at least one person at site. We foresee on-site shifts at \LNGS\ during the commissioning and initial calibrations, but once detector operations have stabilized over an extended period of time, on-site shifting will be transformed into fully-remote data taking operations. Some of the routine calibrations, such as neutron calibration runs will still require on-site shifts.  Responsibilities for shifts will be shared among all groups in the collaboration. Some support for the operations portion of the project has already been awarded, including base-funding from \NSF.  It is expected that the same level of base-funding would continue throughout the operations phase of the \DSks\ detector.

\textbf{\Urania:} The same plant staff personnel that were trained during the commissioning phase of the project will transition directly into operations.  The operations of \Urania\ falls within the overall project scope here since it provides the target for the \DSks\ detector. These staff members will be both the primary \Urania\ full-time plant staff, as well as any \DS\ collaboration members taking part in \Urania\ operations. The methods for supplying \Urania\ with the consumables during operations will be established, including contracts and required delivery schedule. The costs of operations for the \Urania\ facility during the extraction of the \UAr\ for \DSks\ will be supported largely by CFI and PNNL.

\textbf{\Aria:} The \SeruciZero\ prototype will test the purification process of the design, as well as the components that will be implemented in the \SeruciOne\ distillation column. It will start with operations in June 2019 and will lay the groundwork for the standard operating procedures of the \SeruciOne\ column. The \SeruciOne\ plant will run for the time necessary for the production and purification of the \UraniaTotalDSkProduction\ of \UAr\ necessary for filling the \DSks\ detector, and therefore falls under the scope of this project. The \SeruciOne\ plant is estimated to be able to chemically purify the \UAr\ at a rate of \AriaChemicalRate, with operations starting in 2021. Hence, \Aria\ will be ready to receive the \UAr\ from \Urania\ by the time extraction starts, and the combination of the two will be able to deliver all the necessary \UAr\ within the project schedule. 

We have secured some funds for the operation and maintenance of \Aria. We are in the process of receiving another award from the Regione Autonoma della Sardegna, which will complete the funds required.  Operations costs have been calculated assuming the prototype column will operate for \AriaSeruciZeroOperationPeriod\ and \SeruciOne\ will operate for a period of about \AriaSeruciOneOperationPeriod.  In order to achieve the design goal, the \SeruciZero\ prototype column and the \SeruciOne\ column need to be operated \SI{24}{\hour\per\day}. The facility will be operated on shift by crews with \numrange{2}{3} people per shift, \SI{8}{\hour} per crew. The crew will include staff personnel from \INFN\ and temporary contract workers. Carbosulcis employees will also be involved as necessary to ensure the safety of operations.

\subsection{Project Close-out Plan}

\textbf{\DSks:}  Construction of the facilities in Hall C of \LNGS\ will be completed by the middle of 2021.  The cryogenic system of the inner and outer detector will be completed and tested by the middle of 2021. The components and subsystems for the inner and outer detector will be fabricated, assembled and delivered to \LNGS\ by the middle of 2021. \DSks\ Photodetector Modules (PDMs) will be assembled at \NOA\ and intensively tested at at various partnering institutions. The electronics, DAQ, and slow control systems will be partially commissioned before their delivery to \LNGS. Several tests of the \DSks\ TPC inside a test dewar are planned prior to its installation inside the outer veto detector. The tests will include: assembly procedure, mechanical stability of the TPC and its hanging structure, detailed characterization of the \SiPMs, leak check of the sealed TPC vessel, HV stability, and TPC operation in \LAr. The time anticipated for the tests of the inner detector inside the test dewar, assembly, installation and commissioning of the inner TPC and outer veto detector is 6-12 months. Commissioning is the final stage of the project and it will start when both the inner and outer detectors are installed in the \AAr\ cryostat and the cryostat is sealed. We foresee a period of up to 6 months for simultaneous filling of the inner and outer detectors with \UAr\ and \AAr, respectively. The project will be finalized with initial calibrations of both the inner and outer detectors to fully characterize their performances such as leveling of the liquid surface for the inner detector, light yield calibrations for both the inner and outer detector and electron drift life-time in the inner detector. After this point, normal data taking operations will begin.

\textbf{\Urania:} The \Urania\ sub-project will be closed-out when the \Urania\ plant has produced the \UraniaTotalDSkProduction\ of \UAr\ required for \DSks\ and it has been shipped from the extraction site in Cortez, CO, USA to the Aria site in Sardinia, Italy. It is anticipated that the \Urania\ facility will continue to operate after this project, since it represents the only established source of \UAr\ in the world.  It is likely that the \GADMC\ will take over operations with a new project that will be put in place for the \Argo\ experiment.  In this case, the close-out of the \Urania\ sub-project will be in the form of a hand-off of the facility responsibility and operations to the new project team.  The agreement between the current project team and the project team that will take over the \Urania\ facility will be worked out during the final year of this project, with the expectation that the new project team would come on board during the final 3 months of this project in order to be trained on how the facility had been operating up to that point.  It is likely that many members of the current project team would also be members of the new project team, in this case the transition to the new project would be seamless.

\textbf{\Aria:} The close-out of \Aria\ in relation to this project will occur when \SeruciOne\ has chemically purified all \UraniaTotalDSkProduction\ of \UAr\ for \DSks.  The \Aria\ facility will then be transferred to the next project, in similar fashion as to the method of transfer for \Urania.  The current project team members who have \Aria\ related responsibilities will stay on board for the next project that will use the \Aria\ infrastructure, ensuring a seamless closeout.
%The cost of DArT is about 500k€, of which half is funded by RAS and the rest from foreign agencies such as Spain, Switzerland and Canada.Other funds for the columns assembly were given by RAS directly to the  Carbosulcis mining company, in excess of a 1 M€, which performed a lot of civil engineering work for both Seruci-0 and Seruci-1, and most notably the 12 months long complete reclamation of the 350m well and the coring activity in the well, in order to verify the consistency and the nature of the walls where the anchoring of the whole column.


