%---
\section{Project Manangement Controls}

%---
\subsection{Project Management Control Plan}

The \GADMC\ has a well-established project management scheme that includes a top-down flow of information, stemming from the Institutional Board. This board consists of one representative from each of the participating institutions and is responsible for the governing of the project.  Outside of this board, the project contains Work Groups and committees which report the progress of related tasks to the governing boards.  Decisions on the management structure are made within the Institutional Board. 

The cost estimating plan and the \WBS\ are used to establish the roles and responsibilities within the collaboration.  The cost model data and the resource loaded schedule allow for the implementation of the Earned Value Management System (EVMS) methodology, integrating cost, schedule, and scope.  Risks will be recorded in a Risk Registry to record uncertainty that can affect cost, schedule, and/or performance.  All risks that can impact the program will be reported directly to the management for mitigation.

%---
\subsection{Earned Value Management System (EVMS) Plan}
The project has adopted the Primavera P6 project management software, which will be utilized for the tracking of the project with an earned value approach and for producing necessary reports used to assess the progress of the project.  The EVMS will allow the accurate forecasting of potential performance problems and the efficient tracking of project progress.  Project tracking will be carried out by the implementation of the project plan, a valuation of the planned work, and the pre-definition of metrics which quantify the accomplishment of work. The input to the EVMS will be the resource loaded schedule, the institutional accounting information, and the output of the risk analysis.
