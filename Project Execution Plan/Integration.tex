%---
\section{Integration and Commissioning}

The three \DS\ sub-projects are developing their own integration and commissioning plans with specific criteria for approving operational readiness. The final plans will be vetted and approved by the Technical Board and, when appropriate, by an external review committees. Clear specifications are defined for the acceptance of any vendor-provided sub-systems that integrate into the project.


%---
\subsection{Integrations and Commissioning Plan}
\textbf{\DSks:} The \DSks\ detector integration plan is incorporated in the WBS. Each sub-project will be tracked by the Technical Coordinator and Project Leader to ensure timely completion. The Integration group and the Infrastructure group will handle issues related to the integration of the \DSks\ detector into Hall C of LNGS. 

\textbf{\Urania:} Princeton University and the Kinder-Morgan CO2 Company have executed agreements to extract the \UraniaTotalDSkProduction\ of \UAr\ from the feed gas stream. Currently, the design and skid mounted construction of the \Urania\ plant is in a tender process in Italy. A contract with the selected company from the tender process is expected to be in place by November 2019. During  the \Urania\ plant construction, the \Urania\ site will be prepared with all required infrastructure. This includes all permits required for the \Urania\ plant installation and operations, electricity, an adequate foundation (as defined by the \Urania\ plant design), communication equipment,  and arrangements for delivery of consumables (e.g., liquid nitrogen). Once the \Urania\ plant is delivered and installed, final gas connections will be made, feed gas flow will start, and commissioning will begin.

Commissioning will be led by the collaboration under the guidance of the company that designed and built the Urania plant. Staff will be hired specifically for the integration and commissioning, and will then transition to extraction operations. In addition to these primary \Urania\ full-time staff, members of the collaboration will take part in the commissioning. The commissioning will include a full leak-check of the plant, establishing the PLC communications between \Urania\ and the Kinder-Morgan facility, cooling down the plant, and finally flowing gas. Each processing set of the plant will be optimized for maximum final purity and throughput. The commissioning will also be used to train staff for operations after the close out of the project.

\textbf{\Aria:} The \Aria\ plant is under construction at the Monte Sinni mine of the Carbosulcis company in Sardinia, Italy, that recently ended its cycle of coal extraction. The availability of an unused well in the mining site of Seruci, now completely refurbished and outfitted to support the \Aria\ sub-project, will allow the construction of the plant with minimal environmental impact and relatively low cost.  The \SeruciOne\ cryogenic distillation column is about \AriaSeruciHeight\ high and is composed of \AriaCentralModulesNumber\ identical intermediate modules each \AriaCentralModulesHeight\ tall, as well as a top condenser and a bottom reboiler module.  Once completed, it will be the highest distillation column in the world.  Authorization for the installation and commissioning of \SeruciOne\ has already been granted (an excerpt of the original document in Italian and it translation into English are included as Supplementary Documents.)  The construction of the \SeruciOne\ column was preceded by the prototype \SeruciZero\ column, a \AriaSeruciZeroHeight\ tall prototype column composed of one intermediate module, a top condenser, and a bottom reboiler.  The construction of \SeruciZero\ is nearly complete, and operations will start in June~2019.  The column will purify \UAr\ at a rate of \AriaChemicalRate, inline with the \UraniaUArRate\ production of \UAr\ from \Urania.  The plant will be assembled, tested, and operated by a team of institutions including Princeton, \INFN, \CERN, Universit\`a degli Studi di Cagliari, \FNAL, and Carbosulcis.


%---
\subsection{Acceptance/Operational Readiness Plan}
\textbf{\DSks:} Operational readiness requirements will be defined by the TB for each of the sub-systems that integrate into the \DSks\ detector. The responsible Work Groups will asses their systems readiness. The \DSks\ detector will be operational when these sub-systems are fully integrated and the detector is operating sufficiently well to take science quality data.

\textbf{\Urania:} The \Urania\ plant will be ready for extraction operations when production of \UAr\ with a \UraniaArFinalPurity\ chemical purity at a rate of \UraniaUArRate\ has been demonstrated, for a period of seven consecutive days, as spelled out in the tender requirements. Operational readiness will require fully trained staff that will transition to operations as the plant shifts from the commissioning phase to the operation phase. Plans for supplying consumables to the plant and for transporting the pure \UAr\ from \Urania\ are under development. Vendors have been identified and the plan will be in place at the onset of the operation phase.

\textbf{\Aria:} Initial quality and acceptance tests have already been performed on the \SeruciOne\ modules, including helium leak tests at the production site for all \AriaCentralModulesNumber\ central modules, as well as the top and bottom modules, leak tests at CERN for all of the  modules, cryogenic test of bottom module, and an \xr\ weld tests during the installation of \SeruciZero.  Before operation, a final global leak test will be performed.  \SeruciZero\ will inform the definition of the most appropriate processing parameters for the operation of the column and will train the operators and technicians involved. \SeruciZero\ is expected to operate for approximately \AriaSeruciZeroOperationPeriod\ for \SeruciZero.  At the end of this activity, the elements of the prototype (column plus auxiliary equipment) will be recovered and used to complete the installation of \SeruciOne.  The allotted period of operations for \SeruciOne\ is \AriaSeruciOneOperationPeriod.  The \Aria\ project also includes the \DArT\ detector, an experiment to measure the residual radioactivity of the argon after purification with \Aria. 

The funds needed for construction and commissioning of \SeruciOne\ and DArT were obtained from the Regione Autonoma della Sardegna (RAS), \INFN, and the Ministero dell'Istruzione dell'Universit\`{a} e della Ricerca (MIUR). The available funds have been used to procure all the modules for \SeruciOne\ and most components for its operation, all of which are now on site. 