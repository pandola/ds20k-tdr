%---
\section{Introduction}


%---
\subsection{Scientific Objectives}
%Description of the research objectives motivating the facility proposal

The \DSk\ project will be a comprehensive search for high-mass \WIMPs\ using a dual-phase argon time projection chamber (\LArTPC). \DSk\ is planned to run for ten years, accumulating a total exposure of \DSkExtendedExposure, and will have sensitivity to a \WIMP-nucleon cross section of \DSkExtendedSensitivityOneGeVUnit\ (\DSkExtendedSensitivityTenGeVUnit) for \ \WIMPMassOneTev\ (\WIMPMassTenTev) \WIMPs.  Within this exposure, we expect \DSkNuInducedBackgroundExtendedExposureUnit\ induced by high-energy atmospheric neutrinos recoiling from argon nuclei, making \DSk\ the first experiment capable of exploring the neutrino background that defines the so-called ``neutrino floor.''  \DSk\ will also be able to perform a high precision measurement of neutrinos in the event of a galactic supernova burst.

In order to achieve this sensitivity, \DSk\ will be constructed from ultra-low background materials and employ techniques to actively identify background events. The scintillating \LAr\ target used in the \TPC\ will be low-radioactivity argon extracted from an underground source.  This scintillation light will be detected by cryogenic silicon photomultipliers (\SiPMs) that have been designed to maximize blue photon detection efficiency while minimizing background. The \LArTPC\ will be surrounded by an active \LAr\ veto housed inside of a membrane cryostat, a technology developed at \CERN\ for the \pDUNE\ experiments. The design of the \LArTPC\ and the membrane cryostat veto are scalable, making \DSk\ a crucial step towards a future, several hundred tonne \LAr\ detector.

%---
\subsection{Scientific Requirements}
\label{sec:Introduction-ScientificRequirements}
%Comprehensive statement of the Requirements Matrix / Key Science Requirements to be fulfilled by the proposed facility (to the extent possible identifying minimum essential as well as desirable quantitative requirements), which provide a basis for determining the scope of the associated infrastructure requirements

The scientific requirements for the \DSks\ project are driven by the need to reduce and/or actively discriminate background events originating from naturally occurring radiation to a level below \BackgroundFreeRequirement\ in the full exposure of \DSkExtendedExposure, thereby maximizing the discovery potential of the experiement. The major scientific requirements are:

\begin{asparaenum}

\item[\bf The procurement and purification of low-radioactivity argon:]
Argon extracted from the atmosphere (\AAr) contains \ce{^39Ar} with a specific activity of \DSfAArArThreeNineActivity, leading to an unsustainable pile-up rate in a detector the size of \DSk. The \DSf\ detector demonstrated that argon extracted from an underground source (\UAr) has an \ce{^39Ar} concentration 1400 times lower than that of \AAr\ and is suitable for use in a multi-ton \LArTPC. The procurement and purification of \UraniaTotalDSkProduction\ of low-radioactivity \UAr\ argon is a pre-requisite for the \DSk\ project and will be carried out by the \Urania\ underground argon extraction plant and the \Aria\ cryogenic distillation column.

\item[\bf The manufacture of low-background, large-area \SiPM\ photosensors:]
The high cost, unreliable performance at \LArNormalTemperature, and intrinsic radioactive contamination of \PMTs\ makes them unsuitable for use in \DSk. Instead, \DSk\ will rely on silicon photomultiplier (\SiPM) based photodetector modules for light detection. In order to meet the event reconstruction requirements of \DSk, they must have an active area of \DSkPdmAreaStd, photo-detection efficiency better than \DSkPdmPDESpecification, the ability to detect single photons with a signal to noise ratio better than \DSkTileChargeSNRSpecification, and a timing resolution of \DSkTileTimeResolutionSpecification. They must also contribute much less than the \BackgroundFreeRequirement\ allotted for the overall background in the full \DSkExtendedExposure\ exposure due to their own intrinsic radioactivity.

%The use of ultrahigh-purity \PMTs\ for the \GADMC\ program is not possible, in consideration of their cost per unit area, poor and unreliable performance at \LArNormalTemperature, and intrinsic radioactive contaminations.  The \GADMC\ and its progenitor \DS\ Collaboration abandoned in~2015 the consideration of \PMTs\ for its future experiments and placed a strong focus on the in-house development of \SiPMs.  Four years later, we have demonstrated the full range of possibilities enables by \SiPMs.  We have demonstrated the ability to build single channel assemblies of \SiPMs\ with area of \DSkPdmAreaStd, outstanding \PDE\ (photon detection efficiency), and ultra-low background.  Design of \SiPMs\ is in house, and performed at Fondazione Bruno Kessler (\FBK), a public research Institution based in Trento, Italy.  Production of \SiPMs\ will take place at a silicon foundry selected by \INFN\ through a public tender.  Thanks to the strong success of this line of R\&D, Regione Abruzzo and the Italian Government funded a special facility, the {\it Nuova Officina Assergi} (\NOA), for the large-throughput assembly of \SiPMs\ in photodetector modules (\DSkPdms) respecting the technological requirements of \DSks.

%\item[\bf Strong discrimination of \ERs\ in favor of the unambiguous selection of \NRs:]
%The portion of the space parameter for a \WIMP\ discovery ``easy and within reach'' has been already explored to no avail.  No dark matter signal (other than the signal reported from the \DAMA\ Collaboration~\cite{Bernabei:2008ik,Bernabei:2010gs,Bernabei:2017kc}, which is not supported so far by independent verifications) has been reported.  The exclusion region is that defined by the recent observation reported by the \XENON\ Collaboration and showing no clear excess for the exposure of \SI{1}{\tonne\year} in presence of \SI{14}{\ev} of background~\cite{Aprile:2018ct}.  The goal of \DSk\ and \Argo\ is to discover dark matter.
%
%With dark matter interactions very rare, and the onset of $\nu$-induced nuclear recoils for exposures of \DSkExposure\ and beyond, it is of essence to contain the number of instrumental background interactions to \BackgroundFreeRequirement, so that a positive claim can be made with as few events as possible.  A discovery for dark matter could come at any exposure level, and, as we shall see below, even a low but non-zero instrumental background can hinder the task of discovering dark matter.  Before going into the details of how backgrounds affect sensitivity, it is first important to draw a firm distinction: it is statistically much easier to exclude a region of parameter space than it is to make a detection.  Typical exclusion relies on a standard of \SI{90}{\percent} likelihood, while a positive detection requires a \SI{5}{\sgm} standard: this is equivalent to saying that, once a reliable background model for the experiment has been built, the observed signal has a probability of \num{3E-7} of being an artifact of statistical fluctuations in the background.  A signal \SI{3}{\sgm} above the null hypothesis has a probability of \num{E-3}, and is generally referred to as ``evidence'', while a \SI{1}{\sigma} deviation, with a probability of \num{0.16}, is ``uninteresting''.  Fig.~\ref{fig:Discovery-SigmaLin} and Fig.~\ref{fig:Discovery-SigmaLog} show the number of \WIMP-like events an experiment would need to observe in order to reject the null hypothesis to \SI{5}{\sgm} as a function of the number of expected background events, when the background model predicts the background with \SI{100}{\percent}, \SI{10}{\percent}, and \SI{1}{\percent} uncertainty.  As is evident from the figures, there is fairly little difference between \SI{10}{\percent} and \SI{1}{\percent} uncertainty. This results from the uncertainty in the model being small enough, so that the Poisson fluctuations dominate the statistics.  When the model predicts \num{0} or \num{0.1} background events, just \num{5} \WIMP-like events can be statistically significant enough to claim a detection at the \SI{5}{\sgm}.  However, if an experiment expects to see even one background event, the number of \WIMP-like events needed for a discovery goes up to \numrange{10}{15}, depending on the uncertainty in the background model.  This means that a dark matter experiment expecting \num{1} background event will need up \numrange{2}{3} times more \WIMP\ interactions to claim a detection, compared to an experiment which plans to run background-free.  The requirement becomes increasingly more restrictive if the expected number of background events is greater than \num{10}.  More details of this calculation can be found in~\cite{Westerdale:2016ub}.
%
%The use of Pulse Shape Discrimination (\PSD) as the main method to suppress the background and achieve zero-background conditions was demonstrated first by \DSf, reaching the level of one part in \DSfAArROIEventsNumber~\cite{Agnes:2015gu}, and later by \DEAP, reaching levels of better than one part in \DEAPPSDRejection~\cite{Amaudruz:2018gr}, for the suppression of \ERs\ in favor of the unique selection of \NRs.  We believe that this characteristic of \LAr, coupled with the sharp spatial resolution of \LArTPCs, is unique in fostering the ability to produce large-scale \WIMP\ searches containing background from instrumental sources below the \BackgroundFreeRequirement\ level required for an optimal discovery program.
%
%\begin{figure*}[h!]
%\centering
%\includegraphics[width=0.8\columnwidth]{./Figures/Discovery-5sigmaLin.pdf}
%\caption[Number of dark matter-like events needed to claim a WIMP observation (linear-linear scale).]{Number of dark matter-like events needed to claim a WIMP observation at the \SI{5}{\sgm} level, based on the predicted background rate of the experiment, in a linear-linear scale.  Solid lines show the number of dark matter-like events needed, including backgrounds, while dashed lines show the number of dark matter events after subtraction of the expected background.  Blue, green, and purple curves were made assuming uncertainty on the background model of \SI{100}{\percent}, \SI{10}{\percent}, and \SI{1}{\percent}, respectively.}
%\label{fig:Discovery-SigmaLin}
%\end{figure*}
%
%\begin{figure*}[h!]
%\centering
%\includegraphics[width=0.8\columnwidth]{./Figures/Discovery-5sigmaLog.pdf}
%\caption[Number of dark matter-like events needed to claim a WIMP observation (log-log scale).]{Number of dark matter-like events needed to claim a WIMP observation at the \SI{5}{\sgm} level, based on the predicted background rate of the experiment, in a log-log scale.  Solid lines show the number of dark matter-like events needed, including backgrounds.  Blue, green, and purple curves were made assuming uncertainty on the background model of \SI{100}{\percent}, \SI{10}{\percent}, and \SI{1}{\percent}, respectively.}
%\label{fig:Discovery-SigmaLog}
%\end{figure*}
%
%\item[\bf Presence of neutrino-induced \NR\ signal:]
%With the nuclear recoil energy thresholds needed to achieve the excellent electron recoil rejection in \LAr\ from pulse shape discrimination, only atmospheric neutrinos and the diffuse supernova neutrino background are energetic enough to produce nuclear recoils in the \WIMP\ region of interest.  Individual nuclear recoils from coherent scattering of neutrinos from the argon nuclei in the target are indistinguishable from \WIMP-induced nuclear recoils.  We foresee \DSkNuInducedBackgroundUnit\ from this source in the full \DSkExposure\ \DSk\ exposure~\cite{Billard:2014cx} (\DSkNuInducedBackgroundExtendedExposureUnit\ in the extended \DSkExtendedExposure\ exposure).  Though these events are background to the \WIMP\ search, they are also an as-yet-unobserved physics signal.  We refer to other, reducible backgrounds as ``instrumental".

\item[\bf The ability to veto neutron backgrounds:]
Neutrons elastically scattering from argon nuclei are indistinguishable from \WIMPs. The background rate from all sources of neutrons, including radiogenic neutrons from ($\alpha$,n), fission decays in the TPC materials, and cosmogenic neutrons, must be suppressed far below the \BackgroundFreeRequirement\ allotted for the overall background in the full \DSkExtendedExposure\ exposure. The total number of neutrons entering the detector will be reduced by the careful selection and screening of detector construction materials. Neutron events will be rejected by fiducializing the sensitive argon volume and by tagging events with multiple interaction sites, techniques afforded by the sub-centimeter position resolution of the \LArTPC~\cite{Agnes:2015gu}.  The residual neutron rate will be measured and efficiently rejected using a veto detector consisting of two volumes of \AAr\ separated by a thin layer of \ce{Gd}-doped ultra-high purity acrylic (poly(methyl methacrylate), \PMMA) and instrumented with \DSkPdms.

%The effectiveness of this strategy has been validated through a detailed Monte Carlo simulation using  \Geant- and  \FLUKA-based codes~\cite{Bellini:2013kr,Empl:2014ih}.

%Driven by the need to mitigate the environmental impact due to operations in the underground \LNGS, we switched to a veto detector based on the use of a large \pDUNE\ cryostat.  The \LArTPC\ will be surrounded by a bath of \pDUNELArMass\ of liquefied \AAr.  Two buffers of \AAr, separated by a thin layer of \ce{Gd}-doped \PMMA\ (ultra-high purity acrylic) will be instrumented with \DSkPdms\ and will operate as anti-coincidence scintillator detectors, to suppress signals from neutrons and guarantee compliance with the requirement of keeping the overall instrumental background below \BackgroundFreeRequirement.  The anticipated performance of the veto detector in rejecting radiogenic neutrons was established by developing detailed Monte Carlo simulations based on the \Geant-based \GFDS\ code.  The determination of the performance of the \DSks\ veto in rejecting cosmogenic neutrons was determined via \FLUKA-based simulation codes.  These codes are the direct output of an effort lasting two decades for the direct measurement of the cosmogenic backgrounds in Borexino at the same \LNGS\ depth of \DSks\ and for the detailed comparison of the Borexino measurement of neutrons multiplicity and lateral range (the best proxy for energy) with predictions from the same \FLUKA-based codes~\cite{Bellini:2013kr,Empl:2014ih}.


\item[\bf The ability to reject electron recoil backgrounds:]
The rate of electron recoil backgrounds misidentified as nuclear recoil events must be kept well below the total background of \BackgroundFreeRequirement\ allotted for the the full exposure of \DSkExtendedExposure. This will be accomplished by the careful selection and screening of detector materials and by the use of pulse shape discrimination (\PSD) to identify electron recoil events. The \PSD\ performance is heavily dependent on the efficient detection of single photons and is the chief driver of the performance requirements for the photodetectors. The \DEAP\ experiment has used \PSD\ to suppress electron recoil events in a \LAr\ detector to better than a factor \DEAPPSDRejection~\cite{Amaudruz:2018gr}. 

%Several classes of electron recoil backgrounds must be considered:
%
%\begin{compactitem}
%\item[\bf Solar Neutrino-induced Electron Scatters:]
%\hfill \break\ Electron scatters from \PP\ neutrinos occur at a rate of \DSkROIPPRate\ within the region of interest (ROI), giving \DSkROIPPUnit\ in the \DSkExposure\ exposure.  \PSD\ is more than sufficient to reject the \PP\ neutrino background.
%
%\item[\bf \ce{^238U}, \ce{^232Th}, and Daughters:]
%\hfill \break\ Radon produced in the  \ce{^238U} and \ce{^232Th} decay chains may dissolve in the \LAr\ and enter the TPC.  The most important background contribution from this contamination comes from $\beta$ decays of the radon daughters whose spectra fall partly within the \DSk\ ROI.  The \ce{^222Rn} specific activity in \DSf\  was measured to be below \DSfRnTwoTwoTwoSpecificActivityLimit, and in \DEAP, thanks to a reduction in the surface-to-volume ratio, was below \DEAPRadonLevel~\cite{Amaudruz:2018gr}.  The \ce{^222Rn} concentration in \DSk\ is expected to be much lower than the upper limit obtained for \DEAP\ due to a further reduction in the surface-to-volume ratio and expected improvements in the cryogenic systems.  However, even if the \ce{^222Rn} contamination remains at the \DEAP\ upper limit,  only \DSkROIBiTwoOneFourRate\ from \ce{^214Pb} $\beta$ decays would be expected in the \DSk\ ROI, giving \DSkROIBiTwoOneFourUnit\ in the total \DSkExposure\ exposure.  Once again, \PSD\ is more than sufficient to exclude this source of background.
%
%\item[\bf \ce{^39Ar}:]
%\hfill \break The \DEAP\ experiment recently published results from a \DEAPExposure\ exposure of \AAr~\cite{Ajaj:2019wi}, which is comparable to the number of \ce{^39Ar} decays expected in a \DEAPUArBackgroundFreeExtrapolatedExposure\ exposure of \UAr, demonstrating \PSD\ is sufficient to reduce this background in \DSk.
%
%\item[\bf \ce{^85Kr}:]
%\hfill \break\ Recent data from \DSf\ show that the \UAr\ contains \ce{^85Kr} at a specific activity of \DSfUArKrEightFiveActivity, a rate comparable to the \ce{^39Ar} activity.  The \ce{^85Kr} in \UAr\ very likely comes from atmospheric leaks during the \UAr\ collection and purification (or, much less likely, from deep underground natural fission processes).  No attempt was made to remove \ce{Kr} from the \DSf\ \UAr\ target, simply because the presence of \ce{^85Kr} was not expected at the time of purification.  For the \DSk\ target, it is expected that the \Urania\ \UAr\ extraction plant will be able to reduce the \ce{^85Kr} to a level resulting in a specific activity much less than the residual \ce{^39Ar}.  If for some reason the \Urania\ plant is unable to reduce the \ce{^85Kr} to the desired level, \ce{Kr} in \ce{Xe} has been reduced by a factor $\sim$1000 per pass by cryogenic distillation~\cite{Wang:2014bk}, which should be even better for \ce{Ar}.  Calculations show that the \Aria\ cryogenic distillation column can reduce \ce{^85Kr} by a factor of more than \AriaChemicalPerPass\ per pass, making this source of contamination in \DSk\ negligible.

%\end{compactitem}

\end{asparaenum}


%---
\subsection{Facility/Infrastructure}

Several major pieces of infrastructure are needed to meet the scientific requirements of \DSks. These include:

\begin{asparaenum}

\item[\bf \Urania:]
\UAr\ with ultra-low levels of \ce{^39Ar} can be obtained from gas wells in Cortez, Colorado, USA, the only well-characterized source of low-radioactivity argon in the world capable of sustained, high throughput production.  The \Urania\ plant will extract \UraniaTotalDSkProduction\ of \UAr\ from this gas stream as part of the science program of \DSks. \Urania\ is an industrial-scale chemical plant, designed to extract \UAr\ at a rate of \UraniaUArRate, delivering \UAr\ with a chemical purity of \UraniaArFinalPurity.  Construction of the \Urania\ plant is the responsibility of \INFN, which has issued a tender that is nearing the adjudication phase.  The contracted vendor will build the plant and ship it to the Kinder Morgan site in Cortez.  Installation of the plant will be carried out by the \GADMC.  We project that the batch of \UAr\ necessary for \DSks\ will be completed by the middle of 2022.

\item[\bf \Aria:]
The \UAr\ extracted by Urania must be \AriaArFinalPurity\ pure prior to the use of \ce{Zr}0-based getters as point-of-use purifiers during the \DSk\ \LArTPC\ fill. This level of purity will be reached by purification with the \Aria\ facility, specifically with \SeruciOne, a \AriaSeruciHeight\ high cryogenic distillation column for high-throughput, high-resolution chemical purification and active isotopic separation.  \SeruciOne\ can perform the necessary chemical purification of \UAr\ at a rate of \AriaChemicalRate.  \SeruciOne\ was also designed to test the possibility of separating \ce{^39Ar} from \ce{^40Ar} by exploiting the tiny difference in the relative volatility of the two isotopes. When operating at a reduced throughput of \AriaSeruciOneRate, \SeruciOne\ is predicted to deplete \ce{^39Ar} by a factor of \AriaDepletionPerPass\ per pass. By processing the intended target multiple times, \SeruciOne\ could further suppress \ce{^39Ar} by three orders of magnitude. A \AriaNuraxiHeight\ tall prototype column, called \SeruciZero, has recently been completed and will start operations in June~2019, enabling a comprehensive test of the cryogenics and slow controls of \SeruciOne.

\item[\bf \NOA:]
the {\it Nuova Officina Assergi} (\NOA) is a custom-built cleanroom packaging facility for the high-throughput assembly of \SiPM-based photosensors funded by Regione Abruzzo and the Italian Government.  The \NOA\ facility will be located in a cleanroom at the {\it Tecnopolo dell'Aquila} that will be repurposed and made available by the Gran Sasso Science Institute (\GSSI).  \NOA\ will feature state-of the art silicon back-end packaging equipment, including an automated cryogenic wafer probe, a high-throughput automated flip chip bonding machine, and an x-ray inspection machine.  INFN is responsible for the procurement of all major equipment and all major contracts have been already signed. During the production of the DS-20k PDMs, NOA will be staffed with sixteen engineers and technicians. The NOA facility will begin operations at the of 2019 and full production is planned to begin in earl 2020.

\item[\bf \pDUNE\ \LAr\ cryostat:]

Following the approval of the experiment in 2017, LNGS requested that the \GADMC\ reconsider its original approved plan, an organic liquid scintillator veto detector nested inside a water Cherenkov veto detector, in order to minimize any possible environmental impacts of underground LNGS operations. Following this recommendation, the GADMC abandoned its original plan and developed a new solution based on the \pDUNE\ membrane cryostats developed at \CERN. The \LArTPC\ of \DSks\ is placed at the center of the \pDUNE\ cryostat, which will be filled with \pDUNELArMass\ of liquefied \AAr, part of which will be instrumented and serve as a scintillation (and Cherenkov) anti-coincidence veto detector.  The veto detector is composed of an inner volume of active liquid atmospheric argon (Inner Argon Buffer, \IAB) surrounding the \TPC, a passive \ce{Gd}-loaded octagonal acrylic (\PMMA) shell (\GdAS) that completely surrounds the \IAB, and an outer active volume of atmospheric argon (Outer Argon Buffer, \OAB). A copper Faraday cage electrical and optically isolates the \OAB\ from the rest of the argon volume.  The \IAB\ and the \OAB\ will be instrumented with a variant of the \DSkPdms\ designed for the \DSks\ \LArTPC.

\item[\bf Sealed \PMMA\ \TPC:]
The adoption of the \pDUNE\ cryostat meant the vacuum cryostat surrounding the \LArTPC, a major source of radiogenic neutrons, could be replaced by a sealed, ultra-pure \PMMA\ vessel. The vessel serves to contain the \UAr\ and is used as a structural element of the \TPC.

\end{asparaenum}


%---
\subsection{Scientific and Broader Societal Impacts}

The \DSks\ project has a diverse portfolio of scientific and educational broader impacts. The technologies developed for the experiment have applications in medical diagnostics, advanced photon detection, and isotope separation, including the discovery of a novel, commercially viable helium source that today supplies \UraniaHeNationalReserveFractionEquivalentRate\ of the US production; the production of hundreds of tonnes of low-radioactivity \UAr\ for \DSks\ as well as for other technical uses including nuclear test ban verification and radiometric dating; and the development of low-background, large-area, single-photon, cryogenic photosensors.  The planned \Aria\ project for \UAr\ purification may improve the worldwide availability of valuable stable rare isotopes such as \ce{^18O}, \ce{^15N}, and \ce{^13C}, which are used for various medical, industrial, and energy generation applications.  \DSs\ technology has also led to \ThreeDPi, an innovative, patent-pending LAr-based TOF-\PET\ system that can enhance cancer screening sensitivity while dramatically lowering patient radiation dose, under development at Princeton University and Universit\`{a} degli Studi di Cagliari.

Specific education and outreach programs are planned as part of the \DSks\ project with a focus on educating K-12 teachers about basic physics and its relation to dark matter detection, re-starting a summer school experience for high-school and undergraduate students, and giving education and training opportunities to undergraduate students at participating underrepresented-minority serving institutions.

\subsubsection{Scientific Impacts}

\begin{asparaenum}
\item[\bf \SiPMs\ for Medical Diagnostics:] \INFN\ and Princeton University filed patent P137IT00 for \ThreeDPi, an innovative high-definition 3D Positron annihilation vertex imager.  \ThreeDPi\ is designed to overcome the limitations of conventional Positron Emission Tomography (\PET) and Time-Of-Flight PET (\TOFPET) nuclear imaging techniques that are used in the fight against cancer. The market for \PET\ and \TOFPET\ machines is valued in the hundreds of million of dollars per year and is expanding rapidly.  With traditional \PET\ and \TOFPET\ machines, poor time of flight resolution inhibits the 3D position reconstruction of the point where the positron annihilates and the gamma-rays are generated. Therefore, \PET\ and \TOFPET\ machines reconstruct clinical images by 2D tomography, first determining the 2D projections in surfaces perpendicular to the \grs\ line of flight and then combining and fitting the 2D projections obtained for different angles to reconstruct the 3D image.  As a consequence, resolution, contrast, and brightness of clinical images obtained with \PET\ and \TOFPET\ machines are sub-optimal.  A \ThreeDPi\ machine would directly reconstruct the 3D position of individual positron annihilation vertices, event by event. If developed and commercialized, a \ThreeDPi\ machine would turn into an extremely powerful weapon in the fight against cancer.  The key technological advances of \ThreeDPi\ are
\begin{inparaenum}
\item the replacement of crystal scintillators with doped liquid argon, which combines a fast decay time with a very high scintillation light yield;
\item the replacement of traditional \PMTs\ with \SiPMs;
\item and the operation of \SiPMs\ at cryogenic (\LArNormalTemperature) temperature, which dramatically reduces the noise of the \SiPMs.
\end{inparaenum}
The development of \ThreeDPi\ is synergistic with the \DS\ program, both will use cryogenic \SiPMs\ and \LAr\ as a scintillator. A Time-Of-Flight resolution of \ThreeDPiTimeResolution\ is anticipated with \ThreeDPi, enabling the direct reconstruction of the 3D position of each individual positron annihilation vertex. Reaching this long-sought milestone would be transformative for nuclear imaging. The unique capabilities of \ThreeDPi\ would include:
\begin{compactenum}
\item The ability to obtain clinical images with \si{\sim\mm} resolution, allowing oncologists to detect smaller  tumors and more precisely define their position;
\item An improvement in the \SNR\ of commercial \TOFPET\ machines, which would create brighter and higher contrast clinical images and allow a reduction of the radiation doses administered to patients from \SI{10}{\milli\sievert} to \SI{0.1}{\milli\sievert}, thereby decreasing the chances of radiation-induced cancer recurrences. This is of special significance for pediatric patients, who may be subjected to multiple checkup imaging exams in their post-surgery life span;
\item The ability to precisely identify the position of mis-folded proteins (Amyloid-beta, Tau, Alpha-Synuclein, etc.) with long-latency accumulation times, which may be responsible for the onset of neurodegenerative disorders such as Alzheimer and Parkinson diseases.  Very low-dose tracers for these proteins could be used in healthy subjects for preventive and therapeutic treatments;
\item The ability to precisely determine the differential dose delivered to cancerous and healthy cells during hadrotherapy sessions.  This revolutionary capability, described in patent P137IT00, would be achieved by injecting patients with special, cancer cell-targeting tracers prior to the therapy session.  These tracers would be  loaded with stable isotopes selected for their propensity to be activated into positron emitters by the hadron beam delivered during therapy.
\end{compactenum}

\item[\bf \SiPMs\ for Science and National Security:] The development of \SiPMs\ for \DSks\ will drive substantial improvement in this promising technology.  \SiPMs\ have potential impact in many applications over a variety of fields:
\begin{inparadesc}
\item \SiPMs\ could replace \PMTs\ in many particle physics experiments, especially those with the option of operating \SiPMs\ at cryogenic temperatures;
\item \SiPMs\ could also outfit future generation of detectors used for national security;
\item the functional unit of the \SiPM, the \SPAD, is used for fast light detection with broad application, including distance sensing in cars.
\end{inparadesc}
Support for \DSk\ will directly and indirectly support this entire range of activities.

\item[\bf \ce{^4He}:] Helium is a non-renewable resource that is essential for many high-tech industries, for scientific research, and plays a strategic role in the defense and space exploration industry. The Bureau of Land Management (BLM) operates the National Helium Reservoir in Amarillo, 
Texas, the sole governmental helium storage reservoir, and provides on its webpage a list of industrial processes dependent on helium and a list of governmental users that are potentially affected by helium shortages~\cite{USDepartmentofInteriorBureauofLandManagement:2015ui}.  With helium demand growing since the end of the financial crisis, many U.S. universities and laboratories are suffering from rising prices and cutbacks to their helium deliveries.  Since 2008, the \DS\ collaboration has been extracting \UAr\ and measuring the content of helium and \ce{^39Ar} at the Kinder Morgan Doe Canyon \ce{CO2} facility in southwestern Colorado, enabling the start of the first commercial enterprise to extract helium from a \ce{CO2} stream.  Air Products built a helium production plant treating the entire \ce{CO2} stream produced by Kinder Morgan at Cortez~\cite{AirProductsandChemicalsInc:2015tv}.  Production started in \UraniaHeStartDate\ and will replace a \UraniaHeNationalReserveFractionEquivalentRate\ fraction of the declining helium production from the National Helium Reservoir.

\item[\bf \ce{^3He}:] \DS\ collaborators are studying methods to separate \ce{^3He} from massive streams of helium, such as the one that is available at Doe Canyon.  If successful, this effort could help solve future shortages of \ce{^3He}~\cite{Shea:2010vz}.  \ce{^3He} is a very rare element with applications in nuclear fusion and neutron detection. It is also a strategic asset, as its role in dilution refrigerators to reach base temperatures of a few~\si{\milli\kelvin} cannot be easily replaced.  Happer and colleagues developed the technique of lung imaging with hyperpolarized \ce{^3He}~\cite{Happer:1984il}, whose deployment on a large scale, contingent the development of large-scale processes for \ce{^3He} production, could significantly enhance the capability of early detection of a number of lung diseases.

\item[\bf \ce{^37Ar}:] The noble gas radioisotope \ce{^37Ar} is of great interest in the detection of underground nuclear tests.  The production of \ce{^37Ar} via the reaction \ce{^40Ca}($n$,$\alpha$)\ce{^37Ar} has a relatively high cross section and is expected to provide a signature of large numbers of neutrons interacting with the soil~\cite{Riedmann:2011ht}.  As a noble gas, \ce{^37Ar} is expected to migrate to the surface following an underground nuclear test, and with a \ArThreeSevenMeanLife\ mean life, it is sufficiently long-lived to allow time for soil-gas sampling. The chemistry of argon recovery and purification is important for preparing soil-gas samples for treaty verification measurements. This chemistry is synergistic with the challenge of recovering and purifying geologic argon for use in a dark matter detector.  However, \ce{^37Ar} detection is quite challenging, as it decays via electron capture, emitting low-energy Auger electrons and X-rays.  Perhaps the most well-known high-sensitivity measurement of \ce{^37Ar} was performed as a means of measuring the solar neutrino flux incident on the Earth~\cite{Cleveland:1998er}.  The major emissions of \ce{^37Ar} decay are summarized in that work.  Nominally, only two K-capture decay channels are important, both with total energy summing to \ArThreeSevenKCaptureXRaysEnergy.  In the first channel, which has a branching ratio of \ArThreeSevenKOneBR, decays yield only Auger electrons.  This is the easiest decay channel to detect, as the electrons will deposit their energy in a short range in a proportional counting gas held at typical spectrometer pressures.  In the second decay channel, which has a branching ratio of \ArThreeSevenKTwoToFourBR, most or all of the energy is emitted as X-rays.  The efficiency for detecting the full energy of these X-ray emissions varies with the proportional counter operating pressure and geometry. An ideal \ce{^37Ar} measurement would have sensitivity beyond the expected \ce{^37Ar} background rate, \SIrange{1}{200}{\milli\becquerel\per\cubic\meter}~\cite{Riedmann:2011ht} and would provide sufficient capacity to support a worldwide characterization of background \ce{^37Ar}.  Several laboratories  have this capability, notably the University of Bern, but in the last decade, a need has been identified for a U.S.-based \ce{^37Ar} measurement effort with the capacity for many parallel measurements.  Filling this gap was one of the motivations for the recent development of a shallow underground laboratory~\cite{Aalseth:2012jl} and a new proportional counter design~\cite{Aalseth:2013cg} at PNNL.  In this treaty verification application, the chemistry of argon recovery and purification is important for preparing soil-gas samples for measurement.  This chemistry is synergistic with the challenge of recovering and purifying geologic argon for use in a dark matter detector.  PNNL's focus on \ce{^37Ar} for detection of underground nuclear tests~\cite{Aalseth:2013cg}, as well as their focus on \ce{^39Ar} as an age-dating tracer of aquifer residence time, are natural technical complements to a scientific interest in detecting and elucidating the nature of dark matter with a \LAr\ detector.

\item[\bf \ce{^39Ar}:] Internal-source argon gas-proportional counters are used to detect isotopes useful as environmental radiotracers. This is one of the most sensitive methods for the routine assay of challenging radionuclides like tritium~\cite{Theodorsson:1999dn} and \ce{^39Ar}, which are important for the age-dating of water~\cite{Martoff:1992bg}.  As with dark matter detection, the \ce{^39Ar} background in atmospheric-sourced argon becomes an important limit to sensitivity.  The availability of geologic argon from methods developed for \DS\ will extend the reach of these low-level measurements and their application as tracers of environmental processes, such as groundwater residence times, allowing the study of effects occurring on the time-scale relevant for modern climate change patterns.

A U.S.-based effort has been established at \PNNL\ for ultra-low-level proportional counter measurements of argon to support treaty verification and water reserve characterization. Geologic argon samples from the \DS\ collaboration R\&D effort have been used at PNNL to characterize ultra-low-background proportional counters~\cite{Aalseth:2009je,Seifert:2012ip} and estimate the age-dating sensitivity of the technology for modest-sized samples.  The extraction of \UAr\ central to the physics reach of \DSk\ will significantly enhance the ability of researchers to employ \ce{^39Ar} as an environmental radiotracer for hydrologic transport.
 
\item[\bf Ultra-Pure Gases:] Technologies for gas purification are extremely important in some market segments, including the pharmaceutical and semiconductor industries.  The Aria distillation columns have the unique capability of producing high-purity gases, thanks to their thousands of equilibrium stages.  Perfection of this technology would have application in these sectors.

\item[\bf Electronics and Microelectronics:] As the size of transistors in memory and processors continues to shrink, multi-bit flips due to natural and cosmogenic radioactivity is of growing concern.  The radon-abated \CROne\ and \CRH\ clean rooms, with their world-leading \RadonCRHLimit\ limit on \ce{Rn} activity, could benefit studies to characterize and minimize surface $\alpha$-emitter contamination on silicon chips.  A viable measurement technique for surface $\alpha$-emitter contamination that makes use of silicon CCDs was recently introduced in Ref.~\cite{AguilarArevalo:2015hf} and could be useful for this purpose.

\item[\bf IV Generation Nuclear Plants:] The \Aria\ cryogenic distillation columns for isotopic separation will have the ability to separate \ce{^15N} from atmospheric nitrogen gas, nitric oxide or ammonia.  Uranium nitride (\ce{U_n}\ce{^15N_m}) is among the best candidates for fueling IV Generation nuclear reactors due to its superior thermal and mechanical properties~\cite{Zakova:2012dy,Youinou:2014dv,Jaques:2015cw}.  The main drawback of uranium nitride is that it must synthesized from \ce{^15N} with purity greater than \SI{99}{\percent} to avoid neutron absorption on \ce{^14N}. Uranium nitride fuels allow a decreased frequency of refueling shutdowns, higher reactor up-time, and greater economy.  Additionally, the higher density, higher melting temperature, better thermal conductivity, and lower heat capacity of uranium nitride~\cite{Hayes:1990hz,Hayes:1990go} helps  improve the safety margin in reactor design~\cite{Zhao:2014ia}.   The adoption of uranium nitride as the fuel of choice for IV Generation nuclear reactors would create a new market for \ce{^15N}, valued in the hundreds of millions of dollars per year.

\item[\bf Precision cleaning:] The custom precision cleaning facility developed for cleaning \DS\ mechanical parts achieved benchmark levels of cleanliness and cleanliness certification. The European market for precision cleaning is comparatively underdeveloped. This represents a unique occasion for the creation of a technology startup in the L'Aquila district based on a proven technology of interest to the aerospace and pharmaceutical industries.

\item[\bf Nuclear Medicine:] There is great interest in augmented means of production for \ce{^18O} and \ce{^13C}, which are used as precursors of tracer isotopes for tumor therapy, clinical studies, and the development of new drugs.  The development of the Aria cryogenic distillation columns for isotopic separation will lead the way to new methods of producing these important isotopes, improving their availability and lowering costs. \ce{^18O} is a precursor of the positron emitter \ce{^18F}, the core ingredient of \ce{^18F}-FluoroDeoxyGlucose (\FDG)~\cite{Pacak:1969cf}, a glucose analog with replacement of a hydroxil group with \ce{^18F}.  This is the most common radiopharmaceutical used in medical imaging by \PET, \TOFPET, and \PETCT~\cite{Som:1980vv,Kelloff:2005hm}.  \FDG\ also plays an important role in neurosciences~\cite{Newberg:2002hq}.  \ce{^13C} is a marker used in thousands of stable isotope labeled, custom synthesized organic compounds which are used in applications like the \ce{^13C}-Urea breath test, which replaces gastroscopy for identifying infections from Helicobacter pylori~\cite{Graham:1987cy}, and the \ce{^13C}-Spirulina platensis gastric emptying breath test~\cite{Bharucha:2012eu}.  It is also used in fundamental studies in proteomics, carbon fixation, and many other applications.

\end{asparaenum}

\vspace{0.15in}

\subsubsection{Educational Broader Impacts}
The \DSk\ project will make an educational impact on society by improving public understanding of our basic physics research techniques and goals and by offering STEM educational and training opportunities that reach populations of underrepresented students. This is possible thanks to the diverse composition and outreach experience of the collaboration.  

As a central outreach portion of this project, the collaboration plans to run a program for the education and training of students from the argon trail: the Cortez-Durango area, Abruzzo, Italy, and Sardinia, Italy. For this purpose, we plan to re-establish the Princeton-Gran Sasso Summer School of Physics with help from the entire \DS\ Collaboration and in cooperation with the Gran Sasso Science Institute (\GSSI). This Summer School will give rising high-school seniors and rising freshman undergraduate students the opportunity to spend a week each at Princeton and at least one other \DS-collaborating institution in Italy where they will participate in educational and research activities related to the \DSs\ program. 

The focus of the U.S. portion of the collaboration will be the recruitment of students from the Cortez-Durango area, where the \Urania\ plant will operate.  This initiative will be led by faculty members at Fort Lewis College, which, by statute, is ``to be maintained as an institution of learning to which Indian students will be admitted free of tuition and on an equality with white students'' in perpetuity (Act of 61st Congress, 1911). We are requesting funds to support up to five students per year from the Cortez-Durango area to participate in the Princeton-Gran Sasso Summer School of Physics, and we expect significant participation of Native American students.  The students will attend a period of courses at Princeton followed by a research experience at the \Urania\ facility in Colorado, the \Aria\ facility in Sardinia, or \LNGS.  Each student will gain an appreciation for the skills one needs to work in these types of environments and the experience will spur their interest in dark matter physics, engineering, and STEM-related topics in general.  In addition to an educational experience, participants will have the opportunity to meet researchers and professionals in industry, establishing a professional network that can assist them with future career opportunities.  This program will be evaluated on a yearly basis with a survey of the students that gauges their perceived benefits from the program as well as long term follow-ups with past participants to determine any career benefits from their participation. The participation of students from the argon trail in Italy, Sardinia and Abruzzo, will be enabled through separate funds raised from governmental and private sources in Italy.

The \DS\ collaborating institutions serve a diverse group of undergraduate students. Fort Lewis College is a Native American Serving institution and is regionally connected to populations of rural, first-generation immigrants and low-income students. This complements the diverse, urban populations served by our team members from larger institutions, several of which are minority serving (Houston for example is a Hispanic and Asian serving institution). We will engage these students in undergraduate research and, through partnerships within the collaboration, give them access to an interdisciplinary, cutting-edge research program operating at a global scale.

The \DS\ team has experience building compelling narratives around the \DS\ project and the science and engineering challenges involved therein.  These narratives form the basis of a curriculum that will be used by the \DS\ team to conduct outreach to our regional communities. This curriculum may also be shared with K-12 educators as a resource to add locally relevant context to the science they are learning in the classroom.  For example, science teachers in the Montezuma-Cortez, Colorado, school district will be encouraged to use DS-supplied materials based on the argon extraction happening in the region.  Similar activities may take place in the Sulcis-Iglesiente district of Sardinia in the context of the \Aria\ project. 

To date, the \DS\ Collaboration has conducted educational outreach to local public schools (K-12) and engaged high school and undergraduate students in meaningful research related to \DS\ science.  The scale of the \DSks\ project gives us the opportunity to expand upon, integrate, and formalize these efforts so that we can reach a larger, more diverse population of students.  In doing so, we promote the science of \DS\ to a broader audience and build a cadre of scientists and engineers prepared to enter the STEM workforce.