%---
\section{Acquisitions}
 

%---
\subsection{Acquisitions Plans}

Acquisition of the equipment requested in this proposal will take place at the participating institutions, subrecipient to the grant received by Princeton University. Thus, each institution will be responsible for the acquisition of the materials and equipment necessary to fullfill their overall assigned task.  As part of the proposal preparation, each subrecipient institution (University of California at Los Angeles (UCLA), University of California at Davis (UC Davis), University of Houston (Houston), University of Massachusetts Amherst (UMass), University of Hawai'i (Hawai'i), and Virginia Tech (VT)) submitted their Statement of Work, a detailed budget, and a budget justification, approved by both Princeton University's and their own research office prior to proposal submission.  In addition, every subrecipient submitted Princeton University's Subrecipient Commitment form, completed and signed by the authorized official of the institution.  Sub-awards will be distributed according to the budgets submitted for the mid-scale proposal and acquisition plans will follow the budgeted material and equipment allocations. In certain cases, subcontracts will be issued to companies for a specific work related to fabrication or facilities required to carry out the work.  Subcontracts will be issued by the subrecipient institutions that require the work. All subcontract preparation will adhere to the federal policies, and signatures will be carried out by the subrecipient contracts office.  \reftab{AcquisitionsGeneral} lists the participating universities acquisition responsibilities for the \DSk\ detector, while the intended dates for acquisitions related to Urania are listed in \reftab{AcquisitionsUrania}. 

\begin{table}[t!]
\begin{center}
\begin{tabular}{l|c|r|r|r} 
\hline \hline
\multicolumn{1}{c|}{\textbf{Item/Service}}
							& \multicolumn{1}{c|}{\textbf{Institution}}
										&\multicolumn{1}{c|}{\textbf{FY2020}}
													&\multicolumn{1}{c|}{\textbf{FY2021}}
																& \multicolumn{1}{c}{\textbf{FY2022}}\\
\hline
Cryogenics					&Princeton	&\$230,000	&\$180,000	&\$82,000\\
\TPC\ Assays				&Princeton	&\$40,000	&\$40,000	&\$40,000\\  
\LAr\ Purification Getter	&Princeton	&\$250,000	&\$250,000	&\$250,000\\
R\&D Fabrication Charges	&Princeton	&\$150,000	&\$25,000	&\$5,000\\
\hline
\UAr Purification System	&UCLA		&\$129,840	&\$132,360	&\$137,400\\
\hline
High Voltage System			&UC Davis	&\$56,000	&\$95,000	&\\
Field Cage					&UC Davis	&\$22,000	&\$15,000	&\\
Reflector Cage				&UC Davis	&\$14,000	&\$20,000	&\\
\hline 
Fused Silica R\&D			&UMass		&\$11,593	&			&\\
Fused Silica             	&UMass		&\$375,605	&\$187,802	&\\ 
\hline
Veto Camera System			&Hawai'i	&\$15,000	&\$25,000	&\\
Sources [Internal, $\gamma$, $(\alpha,n)$, $(\gamma,n)$]
							&Hawai;i	&\$47,000	&\$90,000	&\$55,000\\
Calibration Insertion Systems
							&Hawai'i	&\$60,000	&\$50,000	&\\
Calibration \$ Integrated Monitoring System	&Virginia Tech	&\$21,000	&\$27,000	&\$24,000 \\
\hline
Wire Winding Machine Construction
							&Houston	&\$30,058	&\$20,058	&\\
\DSps\ Wire Grid			&Houston	&\$35,000 	&			&\\
\DSks\ Wire Grid			&Houston	&			&\$45,000	&\\
R\&D Fabrication Charges    &Houston	&\$15,000	&\$15,000	&\$5,000\\ 
\hline
\end{tabular}
\caption[Acquisitions program.]{Program of acquisitions divided by the main recipient or subrecipient institution with profile of expenditure detailed by year.}
\label{tab:AcquisitionsGeneral}
\end{center}
\end{table}


%---
\subsection{Acquisition Approval Process}

The acquisition process will involve institutional, Collaboration, and funding agency approvals as needed. All institutions will follow federal guidance for large cost acquisition items, requiring three vendor quotes in order to get  the most competitive cost. In addition, the acquisition items that will be used to fabricate parts of the TPC will undergo strict scrutiny for radio purity and the necessary radioassays will be funded from the Princeton portion of the grant or external sources. In case of any conflicting results that may impact the detector sensitivity or schedule, the problem will be discussed within the appropriate Work Group and within the Technical Board, prior to proceeding with the final decision and then enacting the appropriate acquisition plan.  Some of the acquisition items may have long lead times, from several months to a year, so the acquisition plan includes the anticipated lead-time.  Special attention will be paid to subcontracts requiring a request for proposal and notices will be sent to prospective vendors. A typical 1-3 month period will be allocated for collecting proposals and another month planned until the most competitive bids are selected. Thus, large subcontracts will include a 12 month planning period to cover the bidding, selection, and lead time required. 

The largest acquisitions are related to the construction of the Urania plant. Anticipated costs and details of the plant have been developed in close cooperation with the Kinder Morgan CO2 Company, which owns the mineral rights to the gas stream which the \UAr\ is extracted from. This close cooperation is expected to continue with PI A.~Renshaw from University of Houston as the Urania Technical Coordinator serving as the liaison with Kinder Morgan, whose corporate headquarter is also based in Houston.  The bidding process for the electrical work, concrete work, and other civil constructions will be led by the University of Houston, via the request for proposal process, while adhering to the federal guidelines and in cooperation with the Kinder Morgan expectations. 

\begin{table}[]
\begin{center}
\begin{tabular}{l|c|r|r|r} 
\hline
\hline
\multicolumn{1}{c|}{\textbf{Item/service}}
											&\multicolumn{1}{c|}{\textbf{Institution}}
														&\multicolumn{1}{c|}{\bf FY2020}
																	&\multicolumn{1}{c|}{\bf FY2021}
																				&\multicolumn{1}{c}{\bf FY2022} \\
\hline
Site Feed Gas Optimization					&Houston	&\$100,000	&			&\\
Site Preparation, Tools, and Equipment		&Houston	&\$150,000	&			&\\
Inlet and Outlet Piping						&Houston	&\$120,000	&			&\\
Site Electrical Installation				&Houston	&\$650,000	&			&\\
Site Concrete Installation					&Houston	&\$250,000	&			&\\
Site Control Room Installation				&Houston	&\$215,872	&			&\\
Site Plant Readiness Work					&Houston	&			&\$200,000	&\\
Plant Mechanical Erection					&Houston	&			&\$350,000	&\\
Plant Interconnections						&Houston	&			&\$120,000	&\\
Cold Boxes Setup							&Houston	&			&\$350,000	&\\
Plant Interconnection and Instrumentation	&Houston	&			&\$250,000	&\\
Site Control Room Installation				&Houston	&			&\$215,872	&\\
Plant Commissioning Equipment and tools		&Houston	&			&			&\$100,000\\
Plant Commissioning Liquid Nitrogen			&Houston	&			&			&\$150,000\\
Plant Commissioning Power and Utilities		&Houston	&			&			&\$975,000\\
Cryogenic Storage Vessels					&Houston	&			&			&\$265,372\\
\hline
\end{tabular}
\caption[\Urania\ acquisitions program]{Program of acquisitions for \Urania\ that will be made by University of Houston, one of the subrecipient institutions, with profile of expenditure detailed by year.}
\label{tab:AcquisitionsUrania}
\end{center}
\end{table}