%---
\section{Staffing}
%\cmt{Lead Editor}{Andrea Pocar}
%\cmt{Number of pages:}{1-2}

%The resource loaded schedule that has been prepared takes into account the manpower required for the successful completion of each of the tasks.  While much of the manpower already exists within the research groups and is supported through NSF and other agency grant awards, there is a small amount of staffing that will be requested with this proposal, including some technician positions, as well as training of new students and postdocs to perform project related-tasks.  Based on previous experience within the project, the same methods for staffing the necessary positions will be followed as in the past.  


%---
\subsection{Staffing Plan}
%\cmt{from Large Facilities manual}{Staffing FTE plan, per NSF and other project-specific job categories, over time. Application of indirect cost rates must be articulated in Cost Estimating Plan (CEP) and Basis of Estimate (BOE) per Section 4.2 of this manual.} 

\reftab{Staffing} summarizes the personnel support specifically requested in this proposal. Academic personnel involved in the proposed activities but not listed are supported through other awards.

\begin{table}[t!]
\begin{tabular}{lcccc}
\hline \hline
 {\bf Institution}	& {\bf Position}		&{\bf FY2020}	&{\bf FY2021}	&{\bf FY2022}\\
 \hline 
 UC Davis		&assistant specialist 		&12 			&12 			&12\\
 Hawaii 		&mechanical engineer 		&6				&4				&1.5\\
 				&postdoc					&12				&12				&12\\
 Houston 		&postdoc 					&12 			&12 			&12\\
 UMass 			&PI (summer) 				&0.7			&0.7	 		&0.7\\
 				&off-site postdoc 			&12		 		&12 			&12\\
 				&off-site project engineer 	&12 			&12 			&12\\ 
Princeton		&packaging technician 		&3 				&3 				&3\\ 
				&off-site technician 		&12 			&12 			&12\\ 
 				&postdoc					&3				&3				&3\\ 
Virginia Tech 	&postdoc					&12				&12				&12\\
\hline 
\end{tabular}
\caption[Personnel support summary]{Summary of requested support for personnel.  Figures represent yearly effort expressed in months.}
\label{tab:Staffing}
\end{table}


%---
\subsection{Hiring and Staff Transition Plan}
%\cmt{from Large Facilities manual}{Schedule and requirements for hiring and training staff, including timelines for increasing or decreasing staffing levels. Required qualifications for key staff.}

Five postdocs are included as key members, one each for the Hawai'i, Houston, UMass, Virginia Tech, and Princeton groups. They will be searched for with high priority in the early stages of the period covered by this proposal.  The University of Hawai'i postdoc will focus on the development of the calibrations systems and the software that will be used to simulate the calibration data. The University of Houston postdoc will be involved in the construction of the wire extraction grid for DS-Proto and \DSks\ and will cover important construction-related tasks within the Urania sub-project.  The UMass postdoc will take a leading role in the integration of the photodetector plane with the TPC and, while based at \CERN, will be part of the core team of \DS\ personnel for DS-Proto and general TPC assembly, and will work with the UMass PI on the Urania pre-commissioning site operations.  The Princeton postdoc will focus on the development and characterization of the PDMs.  The Virginia Tech postdoc will be involved with the camera system of the calibrations working group and the development and implementation of the slow controls system for the \DSks\ detector.  

The requested engineering support is as follows.  One full-time, off-site Princeton project engineer will have oversight of the cryogenic and gas handling system.  The Hawai'i engineer will be responsible for the design and construction (design, custom assemblies, enclosures, housing) of the DD neutron generator and the photo-neutron source deployment system.  The Department of Physics at University of Hawai'i has a well equipped shop with CNC capabilities and two technicians are supported full-time by the university.  The PI intends to use their services for all standard parts fabrication, for which no support for these technicians' time is requested. Funding is requested by the Princeton group for support to the SiPM packaging operations and, through a full-time off-site technician, for the \SiPM\ detector assembly in Italy.  The UC Davis assistant specialist will supervise the production of the components for the high voltage systems, field cages, and reflector cages, and help with the testing and final integration of those components into the DS-proto and \DSks\ \TPC.  The specialist will also supervise UC Davis undergraduate students involved in the process.