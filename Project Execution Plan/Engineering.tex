%---
\section{Systems Engineering}
%\cmt{Lead Editor}{Andrea Ianni/Hanguo Wang}
%\cmt{Number of pages}{6-8}

%---
\subsection{Systems Engineering Plan}
Systems engineering responsibility falls within each of the project Work Groups (WGs).  Each WG is responsible for the design of their system and ensuring that it meets all project requirements, interfaces, and can be deployed and operated.  Tests and checks will be performed before a design is considered ready.  Designs that have been finalized and approved by the Technical Board are then subjected to an external review committee, consisting of a few members from the project team and engineers and scientists from national labs that are qualified to assess the system.  It is the job of the project Technical Coordinator to ensure that systems engineering is done in collaboration between WGs and that WGs are aware of all interfaces.


%---
\subsection{Systems Engineering Requirements}
The design of each system should be validated with full prototyping, and if necessary, by stand-alone tests. The requirements for the design's interface with other systems must be met and fully assessed prior to approval.  All risks must be assessed and will be included in the preliminary risk assessment (PRA).


%---
\subsection{Interface Management Plan}
The Management Structure of the \GADMC\ is designed to manage three sub-projects that must ultimately converge to complete the project. The interfaces between \DSk, \Aria, and \Urania\ are under the control of the Spokesperson, Executive Board and Institutional Board and the Technical Coordinators of the sub-projects.


%---
\subsection{QA/QC, Operations, and Facility Divestment Plans}

The Collaboration is developing a QA/QC plan that will tie to the WBS and will define the sub-system specific tests. Plans for QA/QC are put in place during the engineering of each system, and each will be described in the \WBS\ by its own line item.  As the systems are developed, and then fabricated, required tests and checks will be made to ensure that specifications are achieved as defined in the \WBS.

Effective QA/QC is fundamental for the project.  During the development of the final project design, comprehensive Monte Carlo simulations and material screening campaigns will be done to validate design choices. These design choices will be certified following a period of research and development, prototyping, and tests performed under stringent conditions.  During the massive production of \DSkPdms, the \UAr\ extraction in Colorado and the purification of the \UAr\ in \Aria, a specific set of instruments and human resources will be devoted to fulfill the QA/QC of the final products.


%---
\subsection{Concept of Operations Plan}

Each of the three sub-projects, the \DSks\ detector, \Urania, and \Aria, will have their own {\bf Operations Plan (OP)}.  Each OP will take into account the personnel in each sub-project (operations manager, process engineer, operators) that will form an {\bf Operations Group (OG}) and the availability of operating fund from their respective funding agencies. For instance, the Urania project will hire a full-time plant staff through the support of \PNNL\ and Carleton University. This staff will include a plant manager, extraction operations engineer, process engineer, technical engineer, and up to \num{9} plant operators for standard operational shifting.  This OG for Urania will also include the \Urania\ Project Leader, Technical Coordinator, and L1 managers to ensure safe operations of the facility and the overall success of the project.


%---
\subsection{Facility Divestment Plan}

Most of the facilities that will be part of \DSk\ can benefit future applications.  For example, \UAr\ extracted with \Urania\ has application in other particle physics experiments.  The same is true of \Aria, which could be used for the chemical purification of other gasses, the isotopic cryo-distillation of isotopes useful in medical applications (\ce{^{18}O}, \ce{^{13}C}) or for producing fuel for future nuclear power plants (\ce{^{15}N}).  For such facilities, a divestment plan will be managed by a Memorandum of Understanding for the future use of the \DSks\ systems.  For the remaining systems and/or subsystems that will be used by \DSks\ and the joint sub-projects, the divestment plan will take into consideration a plan for 10 years of operations starting from the start of data taking with \DSk.