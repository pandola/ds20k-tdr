%---
\subsection{Data Acquisition}
\label{sec:DAQ}

The current design for the \DSks\ electronics and its data acquisition system (\DAQ) accommodates both the large number of sensors and the long drift-time (expected maximum electron drift time is \DSkDriftTimeBig) of the \LArTPC, as well as the readout of the Veto detector.  The trigger rate during normal dark matter search data taking has three major contributions: background events from detector materials, background events from \ce{^39Ar}, and random triggers.  In order to estimate the first term, the event rate measured in the \UAr\ exposures of \DSfs\ is taken, excluding the contribution from the \PMTs\ and from the remaining \ce{^39Ar} in \DSfs.  This is scaled by the ratio of surface areas of \DSks\ and \DSfs, a factor of \DSkOverDSfAreaRatio, obtaining an expected rate of \DSkTriggerSurfaceRate.  It is noted that these events will be concentrated at the surfaces of the active volume.  

By using \UAr\ and making the assumption of an \ce{^39Ar} reduction factor of \DSkDArThreeNineDepletion\ with respect to \AAr,  there will still be an additional rate of about \DSkTriggerDArThreeNineRate\ generated by \ce{^39Ar} decays, uniformly distributed throughout the active volume.  In summary, events with correlated \SOne\ and \STwo\ signals are expected in \DSks\ at a rate of about \DSkTriggerTotalRateBig.  On the other hand, the average singles rate per channel is dominated by the dark count rate (\DCR) of the \SiPMs.  With the required \DSkSiPMDCRSpecificationBig\ specification, this will imply a singles rate per \DSkPdm\ of about  \DSkTileDCRSpecificationBig.  The event rate in the Veto detector, instead, will be dominated by \ce{^39Ar} decays in the instrumented \AAr\ buffers. 


%---
\subsubsection{General DAQ scheme}

The baseline scheme for the \TPC\ and Veto detector \DAQ\ electronics hardware foresees an optical signal receiver feeding a differential signal to a flash ADC digitizer board that is connected to a large Field Programmable Gate Array (\FPGA).  The digital filtering capability within the digitizer board would allow the discrimination of single photoelectron signals and the determination of the time and charge of the individual channel pulses.  For large and slow signals, such as the \STwo\ ionization-born pulses in the \TPC, the digitizer board would provide a downsampled waveform matching the expected signal bandwidth of a few \si{\MHz}. 

The combination of the data processing will provide the needed data reduction to allow trigger-less operation of the readout for the \TPC.  Data from the Veto and \TPC\ detectors will be transferred to the front-end data processing units where further data reduction will be performed.  Finally the data will be passed to an online event building processor that will select interesting events and write them to permanent storage.

In normal data taking mode, an event could be identified by a coincidence of hits in the \TPC\ within a specified time window.  A coincidence of \DSkTriggerThresholdHits\ hits in \DSkTriggerThresholdWindow\ would result in a random trigger rate well below \DSkTriggerRandomRate.  Nuclear recoil events at the trigger threshold, producing about \DSkWIMPMinPE\ in \DSkWIMPMinPETimeWindow, would result in the collection of \DSkWIMPMinTwoHundredns\ within the first \DSkTriggerThresholdWindow.  Thus, the trigger would be \SI{100}{\percent} efficient for the \WIMP-like signals of interest.  

The event building and software trigger stage is realized with modern commodity \CPUs\ and connected through fast ethernet with the front-end DAQ processors.  Given the low expected rate, the trigger-less option is foreseen to be feasible to implement.  The expected combined event size for the \TPC\ and Veto detectors is projected to be well below \DSkDAQGlobalEventSizeUpperLimit.

Synchronization between the \TPC\ and Veto \DAQ\ is fundamental for the effectiveness of the design, and will be provided and maintained during the data taking.  The clock source of the \TPC\ \DAQ\ will be used as reference and digital signals (like GPS time stamps or trigger IDs) will be generated to uniquely identify each event regardless of the trigger origin or the detector.  A pulsed signal distributed to all the modules will be used to check and correct the alignment of each channel among the \TPC\ and Veto detectors.

The \DAQ\ system will be located in an electronics room which will be placed on the roof of the \AAr\ cryostat.  The environment will allow personnel access while minimizing the length of the optical fibers used to transmit the data from the \TPC\ to the signal receivers.


%---
\subsubsection{Digitizers}

The basic readout element of the proposed \DSks\ \DAQ\ system is a multi-channel board hosting several fast \ADCs\ linked to a large \FPGA\ for digital signal processing.  This will be connected to a host \CPU\ for control, monitoring and data formatting using as an output channel through a \DSkDAQEthernetRate\ ethernet connection to an external computer.  The \ADC\ will have \DSkDAQADCBits\ bits resolution and \DSkDAQADCSamplingRate\ sampling speed.  The board will accommodate one octal fADC chip for each of the \num{8} mezzanine boards, allowing if to handle 64 channels in total with a single board with a VME64~6U form factor.  Data from the \ADCs\ will be sent using high bandwidth JESD interface directly to a large \FPGA.  The board will host a Xylinx made Zync Ultrascale+ with a quad-core ARM Cortex A53 processor.  Recent progress on this development includes the full implementation of the JESD interface and the ability to output data at the maximum specified speed.  Specialized firmware is to be provided by the collaboration for noise filtering, basic data formatting and zero suppression to be implemented in the \FPGA.

The development of this electronics board is an ongoing partnership between  the \CAEN\ Company from Viareggio, Italy, which was selected as the provider of the electronics by the \INFN, and the collaboration.  The basic layout of the boards and the components that will comprise them have been selected and the first prototype will be available for testing in the summer of 2019.  In the meantime, work in close cooperation between the \GADMC\ and the \CAEN\ firmware experts is ongoing to implement the needed signal filtering algorithms.  In parallel, a \DAQ\ system for the initial prototype \TPC\ tests has been deployed at \CERN\ in order to acquire data from up to two \DSkPdm\ Motherboards, amounting to \DSpZeroPdmsNumber\ \DSkPdm\ channels.


%---
\subsubsection{DAQ software}

The Maximum Integrated Data Acquisition System (MIDAS) has been chosen as a framework for developing the \DAQ\ readout and related online control software for the \DSks\ detector.  The \MIDAS\ \DAQ\ package has been used extensively within the DEAP-3600 experiment, and together with the \CAEN\ hardware provides a nice baseline for the digitization and recording of the raw data. A collaboration between the \MIDAS\ and \CAEN\ developers has been established to ensure that the interfacing of the front-end hardware is compatible with the back-end hardware and software that will ultimately compile and write the data to permanent storage.

\MIDAS\ is also well suited for handling the \DAQ\ of the \DSps, for which, at least in the first stage, has a readout strategy and hardware implementation similar to those used in the DEAP-3600 experiment.  The DEAP-3600 hardware is mainly composed of 32xVME-V1720 and 4xVME-V1740 using the proprietary \CAEN\ CONET2 optical link to \num{5} different computers, providing optimum individual link data throughput using the A3818C PCIe interface.  The \DSps\ readout will be, in the first phase, based on \num{5} CAEN VME-V1725 and a custom trigger module developed by the TRIUMF group.  This system has been already installed at \CERN\ in the first part of 2019 in order to provide the \DAQ\ system for the first \DSpZeroPdmsNumber\ channels (\DSpZeroMBNumber\ motherboards) of \DSps.  In a later stage, the readout for \DSps\ will be based on the new digitizer boards from \CAEN, previously described.  Work has started toward designing an evolution of the \MIDAS\ software to accommodate the new hardware boards and provide the needed event building and software trigger capability.

