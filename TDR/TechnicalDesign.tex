%---
\section{Technical Design}
\label{sec:TechnicalDesign}

{\bf\color{red}

Viene descritto lo spazio richiesto presso le aree dei laboratori di superficie e sotterranei.
In questa parte inoltre vengono descritte nel dettaglio tutte le scelte tecniche fatte, la loro implementazione e conseguentemente i sistemi e sottosistemi. In particolare si richiedono le specifiche funzionali e tecniche dell’infrastruttura, degli impianti e delle attrezzature in termini di:
\begin{itemize}
\item opere civili, 
\item impianti elettrici, 
\item HVAC (riscaldamento, ventilazione e condizionamento dell’aria), 
\item acqua di raffreddamento/ deionizzata, 
\item aria compressa,
\item attrezzature in pressione, 
\item sistemi da vuoto e criogenici, 
\item sistemi di sollevamento e trasporto, 
\item supervisione e controllo, 
\item sicurezza, 
\item  IT, 
\item radioprotezione, 
\item meccanica, 
\item elettronica.
\end{itemize}
Si richiede di allegare i documenti tecnici al momento disponibili dalla Collaborazione (esempio: P\&I, schede tecniche, elaborati tecnici, …).
}

\vspace{1cm}

%--
\subsection{Installation Space and Services Needed}
\label{sec:SpaceAndService}
The complex installation procedures will be described in details in Section~\ref{sec:InstallationSequence}, here we give the general overall size of the needed spaces and services to be available in Hall C for a smooth, fast and successful construction of the \DSks detector.

\vspace{}

{\bf DISCUSSION OF THE SABRE ISSUE}

It is to be noted preliminarily that the installation sequence is designed to use all the HALL-C floor from the entrance up to the CTF and other Borexino installations. This arrangement which anticipate vacating the space allocated, as of this writing, to the SABRE POP is optimal from the point of view of simplifying and thus speeding up considerably the installation of the outer cryostat. In addition the same space would be used for installation of counting rooms and other vital equipment for 
\DSk that otherwise would need to be installed on steel structures overhanging the SABRE POP space, complicating the design, cost and construction of those structures.
(to be elaborated further)

\vspace{1cm}

The requirements for the installation of the \DSk detector are then summarized below:

\begin{itemize}

\item Total available surface in Hall C: 14m x 26m (or 2 x 14m x 14m). The surface needed for cryostat installation is 14m x 14m, and additional 14m x 14m for cryostat parts pre-assembly. 
\item Connections panels for construction phase (“quadri da cantiere”): Four 3-phases panels (32A each) for a simultaneous use of welding machines
\item Mono-phase connection plugs of 230 VAC (16A each)
\item Panel 24VAC for lighting inside the cryostat.
\item Pressurized air: oil free, 6.3bar, 300 l/min with 3000 l/min peaks.
\item Availablity of Hall C cranes, in particular:
\begin{itemize}[label=-]
\item Use of the 20+20 tons arch crane for cryostat installation and cryostat roof operations
\item Use of the flat cranes for clean room mounting, test cryostat and TPC mounting operations.
\item Removal of the 5 ton cranes to optimize the operations of the 20+20 tons arch crane.
\end{itemize}
\item Unloading Station availability in the TIR gallery.
\item Gas detection systems (Oxygen detectors) and fire protection systems in Hall C.

\end{itemize}

%--
\subsection{Electrical Power for \DSks Operation}
\label{sec:ElectricalPower}

\newcommand{\TotalPowerConsumption}{\SI{222}{\kW}}
\newcommand{\TotalUPSPowerConsumption}{\SI{85}{\kW}}

A total power of \TotalPowerConsumption  is needed for \DSks:

\begin{itemize}
\item Cryostat (reference as in CERN NP04 cryostat) : 27KW 
\item	DAQ (see Air Conditioning System) 
\item	Ancillaries (Water Plant, Exhaust, Radon Abatement System, Lighting): 50KW 
\item Nitrogen Recovery System: 100KW (connection already there in Borex electric box)
\item Air Conditioning System: 45KW 
\end{itemize}
\vspace{\baselineskip}

In addition a powerful UPS line is needed with a total capability of delivering \TotalUPSPowerConsumption for 
the time needed to put in safe conditions all the delicate equipment: 
\begin{itemize}
\item Cryostat and external cryogenics: 10KW
\item DAQ: 45KW 
\item Gas Panel: 25KW 
\item Control System: 5KW 
\end{itemize}

%--
\subsection{Nitrogen Argon and Water Services for \DSks Operation}
\label{sec:NitrogenArgonAndWater}

For cooling purposes the following water supply is needed:
\begin{itemize} 
\item Cryo-coolers and external cryogenics: TBD 
\item Detector cryogenics: 1.0 mc/h 
\item Air Conditioning System: 45KW (1.5 mc/h)
\item Fan-coil building : to be provided from LNGS services.
\item Radon Abatement System: 0.45 mc/h 
\end{itemize} 

\vspace{\baselineskip}

For Nitrogen and Argon the following is forseen:
\begin{itemize}
\item Use of the \SI{20}{\cubic\meter} Nitrogen LINDE tank placed in the TIR gallery (currently in use by \DSf\ and Borexino 
\item Use of High Purity Nitrogen system
\item Use of the already exiting Nitrogen recovery system. The system is  ready to be used. It is instrumented with two Stirling engines, and with a distribution system for the recovered nitrogen in gaseous phase and for the disposal of the recovered liquid nitrogen in the 20 m3 LINDE tank.
\item  Use of a \SI{10}{\cubic\meter} buffer Argon tank for 5.0 Argon to be used for cryostat filling operations and to be located externally to Hall C, along the TIR gallery with a distribution line serving Hall C.
\end{itemize}


%--
\subsection{Underground Argon Storage and Transport}
\label{sec:UndegroundArgonStorage}
\begin{itemize}
\item The current \DSks\ detector and cryogenic services layout implies the use of the TIR gallery for the storage of depleted underground argon (UAr) arriving from the ARIA plant in Sardinia.
\item The UAr transportation system and containers will also play the role of UAr recovery system, it is therefore foreseen that it is disposed in TIR gallery for the whole experiment lifetime.
\item The dimensions, volumes and operational working points of UAr shipping and storage system will be specified as soon as they will be defined.
\end{itemize}


%--
\subsection{IT Infrastructure}
\label{sec:ITInfrastructure}

Data connection with external would need 2 fibers with 10 Gbps bandwidth (to be updated)

%In order to successfully achieve the installation process, the commissioning and the operations of the DarkSide-20k detector and its ancillary infrastructures at Laboratori Nazionali del Gran Sasso, the collaboration requests the following services to be available in the Hall C:

%Electrical power: 
%⁃  Normal line: 
%	⁃  Cryostat (reference as in CERN NP04 cryostat) : 27KW 
%	⁃  DAQ (see Air Conditioning System) 
%	⁃  Ancillaries (Water Plant, Exhaust, Radon Abatement System, Lighting): 50KW 
%	⁃  Nitrogen Recovery System: 100KW (connection already there in Borex electric box)
%	⁃  Air Conditioning System: 45KW 
%⁃  TOTAL power: 222 KW 
% UPS line: 
%	⁃  Cryostat and external cryogenics: 10KW
%	⁃  DAQ: 45KW 
%	⁃  Gas Panel: 25KW 
%	⁃  Control System: 5KW 
%⁃  TOTAL power: 85 KW 
%
%Cooling water : 
%	⁃  Cryo-coolers and external cryogenics: TBD 
%	⁃  Detector cryogenics: 1.0 mc/h 
%	⁃  Air Conditioning System: 45KW (1.5 mc/h)
%	⁃  Fan-coil building : to be provided from LNGS services.
%	⁃  Radon Abatement System: 0.45 mc/h 
%
%Nitrogen:
%	⁃  Use of the 20 m3 LINDE tank placed in the TIR gallery (currently in use by DS50 and Borexino 
%	⁃  Use of High Purity Nitrogen system
%Nitrogen recovery system:
%	⁃  Already existing and ready to be used. It is instrumented with two Stirling engines, and with a distribution system for the recovered nitrogen in gaseous phase and for the disposal of the recovered liquid nitrogen in the 20 m3 LINDE tank.
%Argon:
%	⁃  Use of a 10 m3 buffer Argon tank for 5.0 Argon to be used for cryostat filling operations and to be located externally to Hall C, along the TIR gallery with a distribution line serving Hall C.
%Data connection:
%	⁃  2 fibers with 10 Gbps bandwidth.
%
%Hall C cranes:
%	⁃  In particular the use of the 20+20 tons arch crane for cryostat installation and cryostat roof operations
%	⁃  the use of the flat cranes for clean room mounting, test cryostat and TPC mounting operations.
%	⁃  removal of the 5 ton cranes to optimize the operations of the 20+20 tons arch crane.
%Services:
%	⁃  Unloading Station availability in the TIR gallery.
%	⁃  Gas detection systems (Oxygen detectors) and fire protection systems in Hall C.
%
%Needs for cryostat construction in Hall C:
%
%	⁃  Total available surface in Hall C: 14m x 26m (or 2 x 14m x 14m). The surface needed for cryostat installation is 14m x 14m, and additional 14m x 14m for cryostat parts pre-assembly. 
%	⁃  Connections panels for construction phase (“quadri da cantiere”):
% Four 3-phases panels (32A each) for s simultaneous use of welding machines.
%Mono-phase connection plugs of 230 VAC (16A each)
%⁃  Panel 24VAC for lighting inside the cryostat.
%⁃  Pressurized air: oil free, 6.3bar, 300 l/min with 3000 l/min peaks.

%Depleted Underground Argon:
%	⁃  The current DarkSide-20k detector and cryogenic services layout implies the use of the TIR gallery for the storage of depleted underground argon (UAr) arriving from the ARIA plant in Sardinia.
%	⁃  The UAr transportation system and containers will also play the role of UAr recovery system, it is therefore foreseen that it is disposed in TIR gallery for the whole experiment lifetime.
%	⁃  The dimensions, volumes and operational working points of UAr shipping and storage system will be specified as soon as they will be defined.
