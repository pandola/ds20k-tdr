%---
\section{Radiation Protection Issues}
\label{sec:RadioProtection}

%{\bf\color{red}
%Vengono presentati e discussi i punti specifici che riguardano la radioprotezione presentando i risultati delle eventuali simulazioni delle radiazioni e la descrizione dei sistemi di rivelazione e delle schermature. Si richiede di integrare queste considerazioni con le valutazioni ambientali, operative e di sicurezza gi\`a validate e da personale esperto.
%}
%\vspace{1cm}

%Detailed simulation results on the shieldings and on the active devices of the DarkSide-20k detector and infrastructure layout including calibration system and source storing, will be added as soon as they will be available.

%Detailed reports and operative procedures will be developed in accordance with the European and Italian regulation and LNGS rules in matter of radio protection (D.Lgs. 230/95 and further modifications).

The use of radiation sources is fundamental for the calibration purposes of the \DSks\ detector.
The management of all the sources will be executed in accordance with the European and Italian regulation in matter of radiation protection (D.Lgs. 230/95 and further modifications and addenda) and LNGS radiation protection internal rules.
The intent to introduce and use any kind of radiation sources will be taken with the consent and the authorization of the LNGS Director and LNGS Qualified Expert in radiation protection (RPE).
As reported in Section~\ref{sec:Calibration} distributed gas sources, gamma and neutron sources will be used.
Gas sources as \ce{^83Rb}/\ce{^{83m}Kr} and \ce{^{220}Rn} will be deployed into the stainless steel gas system. A metal thickness of 1 mm will be enough to shield the radiations. Measurements of the dose rate can be eventually performed. Once the gas will be diffuse in the sensitive volume, the thickness of the surrounding material (acrylic, argon, stainless steel) and the distance from the source, will prevent any possible radiation exposure.
For the external gamma ray sources, gamma radiations of 122 keV for \ce{^57Co}, 356 keV for \ce{^133Ba}, and 662 keV for \ce{^{137}Cs} can be easily shielded by the stainless steel guiding tubes, the argon buffer and by the distance from the sources.
Neutrons from AmC sources with a flux of the order of hundreds of neutron/sec will be shielded by the argon buffer and do not contribute to any dose rate.
Neutron sources as AmBe and YBe emit both neutrons and harder (order of MeV) gamma rays. Neutrons from AmBe are mainly shielded by the argon buffer while gamma's are attenuated and absorbed by the metal of the guiding tubes, argon and metal of the cryostat. For the YBe a dedicated 15 cm tungsten shield will be used to block the gamma rays in addition to the ones used for the AmBe sources.

A detailed report on the operations will be delivered for the use of the neutron sources. In general, procedures for the handling and the deployment of the radiation sources accordingly with the LNGS radiation protection Internal Rules reduce the risk of the exposure for the personnel involved in the operations. 
Considering the radiation types and the shielding materials provided in the detector and taking into account the procedures foreseen in the LNGS radiation protection rules, the personnel handling the required radiation sources will be classified as non-radiation workers and the places close to the detector where the sources will be handled, located and used can be classified as non-radiation areas.
Any other operation potentially involving the operative staff during the handling and use of the sources to unforeseen risk of radiation exposure will be discussed in advance with the RPE and performed only after  agreement with the RPE and the authorization of the LNGS Director.






