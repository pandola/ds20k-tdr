%---
\section{Broader Impacts}
\label{sec:BroaderImpact}

%This collaboration offers a unique set of broader impacts that benefit society in measurable ways.  These benefits include technological advances and specific educational opportunities for underrepresented populations that can increase the diversity of a STEM-competent work force.  We believe the diversity of the partner institutions creates opportunity and enhances the overall success of the program.

%{Full discussion of BI, along with a discussion of student training, increased participation of underrepresented groups, and assessment of those efforts.  Include a description of tangible benefits to the wider U.S. research community (access, data products, technology, etc.), including the anticipated impact of those benefits. }


%---
\subsection{Advances in Technology}

%The technologies developed by the DS Collaboration programs benefit society at large in a variety of ways. Broadly, these technologies fall into categories of medical diagnostics, advances in photon detectors, and isotope separation.

INFN and Princeton University recently filed patent P137IT00 for an innovative, LAr-based, high-definition 3D positron annihilation vertex imager called \ThreeDPi. The \ThreeDPi\ project uses LAr-TPC technology to overcome existing limitations of Positron Emission Tomography (\PET) and Time-Of-Flight \PET\ (\TOFPET). These nuclear imaging techniques are used in the fight against cancer, neurological-imaging, and cardio-imaging.  

The development of \SiPMs\ for \DSs\ will drive substantial improvements to this technology.  \SiPMs\ could replace PMTs in many particle physics experiments, especially those needing to detect photons at cryogenic temperatures, in magnetic environments, and in the presence of strong electric fields. \SiPMs\ could also be used in future generations of detectors for national security purposes. Finally, the functional unit of \SiPMs, the \SPAD, finds application in fast light detectors, such as \LiDaR\ distance sensors in cars.

The \GADMC\ Collaboration will advance techniques for isotopic separation and the collection of rare isotope gases that are important to industry, scientific advancement, and national security.  Examples include \ce{^4He} (and \ce{^3He}) extracted from underground \ce{CO_2} wells at the Cortez facility in southwestern Colorado.  These helium isotopes are essential, non-renewable resources for university research the and high-tech industry.  Helium isotopes play an indispensable role in national defense projects and the space exploration industry.  The \GADMC\ collaboration has already played a major role in improving the US availability of \ce{^4He}.  In conjunction with the \Urania\ project, \DSs\ researchers are investigating methods for separating \ce{^3He} from massive streams of helium, {\it e.g.} the one at Cortez.  If successful, this effort could help solve possible future shortages of \ce{^3He}~\cite{Shea:2010vz}, a very rare isotope that has application in nuclear fusion and neutron detection.

The \ce{^37Ar} radioisotope is of great interest for the detection of underground nuclear tests.  The production of \ce{^37Ar} via the reaction \ce{^40Ca}($n$,$\alpha$)\ce{^37Ar} has a relatively high cross section and is a signature of a large flux of neutrons interacting with soil~\cite{Riedmann:2011ht}.  Being a noble gas, \ce{^37Ar} is expected to migrate to the surface following an underground nuclear test where it can be studied.  The chemistry of argon recovery and purification is important for preparing the soil-gas samples for this type of measurement and has significant overlap with the challenge of recovering and purifying geologic argon for use in a dark matter detector.  

Internal-source argon gas-proportional counters are used to detect environmental radio-tracer isotopes.  One of the most sensitive methods for age dating water relies on assaying challenging radionuclides like tritium~\cite{Theodorsson:1999dn} and \ce{^39Ar}~\cite{Martoff:1992bg}.  As with dark matter detection, the \ce{^39Ar} background in atmospherically-sourced argon becomes an important limit to sensitivity. The availability of low-radioactivity geologic argon from methods developed for \DSs\ will extend the reach of these measurements. Geologic argon samples from the \GADMC\ Collaboration R\&D effort have been recently used at Pacific Northwest National Laboratory to characterize ultra-low-background proportional counter backgrounds~\cite{Aalseth:2009je,Seifert:2012ip}.  The \UAr\ development that is central to the physics reach of \DSs\ will significantly enhance the ability of researchers worldwide to employ \ce{^39Ar} as an environmental radio-tracer for hydrologic transport.

The \Aria\ project may help improve availability and lower the costs of rare stable isotopes such as \ce{^18O}, \ce{^13C}, and \ce{^15N}.  \ce{^18O} and \ce{^13C} are widely used as precursors of tracer isotopes for tumor therapy, clinical studies, and the development of new drugs.  \ce{^18O} is a precursor of the positron emitter \ce{^18F}, the core ingredient of \ce{^18F}-FluoroDeoxyGlucose (\FDG)~\cite{Pacak:1969cf}, a glucose analog with a hydroxil group replaced with \ce{^18F}.  This is the most common radiopharmaceutical used in medical imaging by \PET, \TOFPET, and \PETCT~\cite{Som:1980vv,Kelloff:2005hm}.  \FDG\ also plays an important role in neuroscience~\cite{Newberg:2002hq}.  \ce{^13C} is a marker used in thousands of stable-isotope-labeled, custom-synthesized organic compounds with numerous applications.  Its traceability by nuclear magnetic resonance allows applications such as the \ce{^13C}-Urea breath test, which can replace gastroscopy for identifying infections from Helicobacter pylori~\cite{Graham:1987cy}, and the \ce{^13C}-Spirulina platensis gastric emptying breath test~\cite{Bharucha:2012eu}.  It is also used in fundamental studies in proteomics, carbon fixation, and many other applications.

Uranium nitride loaded with \ce{^15N}, \ce{U_n}\ce{^15N_m}, is among the best candidates for  fueling IV Generation nuclear reactors due to its superior thermal and mechanical properties~\cite{Zakova:2012dy,Youinou:2014dv,Jaques:2015cw}.  The main drawback is that uranium nitride must be synthesized from \ce{^15N} that is more than \SI{99}{\percent} pure to avoid neutron absorption on \ce{^14N}. The  advantage of  uranium nitride fuels is that they require fewer refueling shutdowns, larger up-times, and greater economy.  The  higher density, higher melting temperature, better thermal conductivity, and lower heat capacity~\cite{Hayes:1990hz,Hayes:1990go} of uranium nitride compared with other oxides helps improve the safety margin in reactor design~\cite{Zhao:2014ia}.  The adoption of uranium nitride as the fuel of choice for IV Generation nuclear reactors would create a new market for \ce{^15N} valued in the hundreds of millions of dollars per year.


%---
\subsection{STEM Education and Outreach}

The GADMC collaboration is composed of a diverse group of scientists with significant collective STEM outreach experience. Together, we will continue to advance public understanding of particle astrophysics and dark matter science, taking advantage of the scale and excitement surrounding the \DSks\ experiment to provide STEM educational opportunities that reach underrepresented students.

As a central outreach portion of this project, the collaboration plans to run a program for the education and training of students from the argon trail: the Cortez-Durango area, Abruzzo, Italy, and Sardinia, Italy. For this purpose, we plan to re-establish the Princeton-Gran Sasso Summer School of Physics with help from the entire \GADMC\ Collaboration and in cooperation with the Gran Sasso Science Institute (\GSSI). This Summer School will give rising high-school seniors and rising freshman undergraduate students the opportunity to spend a week each at Princeton and at least one other \DS-collaborating institution in Italy where they will participate in educational and research activities related to the \DSs\ program. 

The focus of the U.S. portion of the collaboration will be the recruitment of students from the Cortez-Durango area, where the \Urania\ plant will operate.  This initiative will be led by faculty members at Fort Lewis College, which, by statute, is maintained as an institution of learning to which Native American students will be admitted free of tuition in perpetuity (Act of 61st Congress, 1911). 
% commented due to the word Indian
%``to be maintained as an institution of learning to which Indian students will be admitted free of tuition and on an equality with white students'' in perpetuity (Act of 61st Congress, 1911). 
We are requesting funds to support up to five students per year from the Cortez-Durango area to participate in the Princeton-Gran Sasso Summer School of Physics, and we expect significant participation of Native American students.  The students will attend a week at Princeton where they will receive classroom instruction in particle astrophysics and get hands-on experience with scientific instrumentation followed by a research experience at the \Urania\ facility in Colorado, the \Aria\ facility in Sardinia, or \LNGS.  Each student will gain an appreciation for the skills one needs to work in these types of environments and the experience will spur their interest in dark matter physics, engineering, and STEM-related topics in general.  In addition to an educational experience, participants will have the opportunity to meet researchers and professionals in industry, establishing a professional network that can assist them with future career opportunities.  This program will be evaluated on a yearly basis with a survey of the students that gauges their perceived benefits from the program as well as long term follow-ups with past participants to determine any career benefits from their participation. The participation of students from the argon trail in Italy, Sardinia and Abruzzo, will be enabled through separate funds raised from governmental and private sources in Italy.

The \DS\ collaborating institutions serve a diverse group of undergraduate students. Fort Lewis College is a Native American Serving institution and is regionally connected to populations of rural, first-generation immigrants and low-income students. This complements the diverse, urban populations served by our team members from larger institutions, several of which are minority serving (Houston for example is a Hispanic and Asian serving institution). We will engage these students in undergraduate research and, through partnerships within the collaboration, give them access to an interdisciplinary, cutting-edge research program operating at a global scale.

The \DS\ team has experience building compelling narratives around the \DS\ project and the science and engineering challenges involved therein.  These narratives form the basis of a curriculum that will be used by the \DS\ team to conduct outreach to our regional communities. This curriculum may also be shared with K-12 educators as a resource to add locally relevant context to the science they are learning in the classroom.  For example, science teachers in the Montezuma-Cortez, Colorado, school district will be encouraged to use DS-supplied materials based on the argon extraction happening in the region.  Similar activities may take place in the Sulcis-Iglesiente district of Sardinia in the context of the \Aria\ project. 

To date, the \GADMC\ Collaboration has conducted educational outreach to local public schools (K-12) and engaged high school and undergraduate students in meaningful research related to \DS\ science.  The scale of the \DSks\ project gives us the opportunity to expand upon, integrate, and formalize these efforts so that we can reach a larger, more diverse population of students.  In doing so, we promote the science of \DS\ to a broader audience and build a cadre of scientists and engineers prepared to enter the STEM workforce.
