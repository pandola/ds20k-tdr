%---
\section{Preliminary Activities Accomplished}
\label{sec:PreActivities}

%{Include relevant activities that have prepared the infrastructure project to be implemented, including identification of the primary scientific, technical and system performance requirements, and associated designs and specifications.  For all proposals, if Conceptual, Preliminary and or Final Designs are available, include them as part of the Special Information and Supplementary Documents section.}

\subsection{\LArTPC\ and \pDUNE\ Cryostat Design}
\label{sec:TPC-Cryostat}

The conceptual design of the \DSks\ \LArTPC\ has been finalized.  Verification of the design will proceed through a staged effort at \CERN, first using a small prototype, \DSz\ (\DSzs), and then the \DSpApproximateMass\ scale \DSp\ (\DSps). The fabrication and construction of \DSzs\ is nearly complete, and the detector will be operated at \CERN\ in the summer of 2019. It will test the first \DSkPdmsFirstBatchNumber~\DSkPdms\ produced by the collaboration and will study the effect of the TPC geometry on the \STwo\ signal generation and detection.  The \DSps\ detector will be equipped with \DSkPdmsSecondBatchNumber~\DSkPdms\ and will be a scaled-down version of \DSks\ \LArTPC, serving as its proof of principle prototype. Work is ongoing at \CERN\ and at collaborating U.S. and Canadian Institutions to complete the final design of the detector.

A team of engineers and researchers, with members from the \GADMC\ Collaboration and from \CERN, is currently co-located at \CERN\ and making rapid progress on the design of the \pDUNE\ cryostat and the integration of the \LArTPC\ and the veto detector into the cryostat.

%---
\subsection{Cryogenics and Gas Handling System}
\label{sec:Cryogenics}

The design of the cryogenics and gas handling system for \DSks\ is complete.  Fabrication has already started at \CERN, with various major components already completed and tested. The documents necessary for certifying the equipment and completing the \CERN\ final safety review are moving in parallel with construction. The final safety review performed by \CERN\ will be satisfactory for \LNGS.


%---
\subsection{Photoelectronics}
\label{sec:PhotoElectronics}

The production of the first set of \DSkSQBPdmsNumber\ \DSkPdms\ is complete and will be used in the upcoming \DSzs\ test. The first opto-link system, which combines an optical driver and an optical receiver board for routing PDM signals out of the TPC, has also been produced and tested.  Production of the second set of 25 PDMs and the second opto-link system is ongoing, with half of the \SiPM\ tiles already mounted and successfully tested at cryogenic temperature.  

In order to accommodate the large volume of \SiPM\ wafers necessary for \DSks, \FBK, who produced all of the \SiPMs\ used for the development of the \DS\ \DSkPdms\, has transferred their \SiPM\ technology to a large silicon foundry, \LFoundry. \LFoundry\ completed their first engineering run in September of 2018, demonstrating the successful implementation of  \FBK's \SiPM\ technology.  A second engineering run devoted to the implementation of Through Silicon Vias (\TSVs) is currently ongoing at \LFoundry. Production of the \SiPMs\ that will equip the \DSkPdmsSecondBatchNumber~\DSkPdms\ for \DSps\ will also be carried out at \LFoundry, with the first batch expected by the end of June 2019.

The construction of \DSks\ will require the production of more than \DSkPdmsNumber\ \DSkPdms\ in 2.5 years.  This can only be accomplished using \NOA, a dedicated silicon packaging facility outfitted with cutting edge equipment and highly trained personnel. Assembly of the \NOA\ facility is underway. The \INFN\ has completed tenders for two major pieces of equipment, a cryogenic wafer prober and a flip-chip bonder, and their delivery is expected in the fall of 2019.  The Collaboration has begun training students and dedicated personnel in a temporary clean-room facility at \LNGS\ until the \DSkPdmsCleanRoomSurface\ clean-room space at the Tecnopolo dell'Aquila, which will eventually host \NOA, becomes available.


%---
\subsection{Underground Argon Extraction and Purification: \Urania}
\label{sec:Urania}

\INFN\ opened a tender for the construction of the \Urania\ underground argon extraction plant in 2018.  The adjudication is expected by July~2019.  All bidders have received the detailed technical specifications provided by \INFN\ in accordance with the \GADMC\ Collaboration requirements, including the extraction rate of \UraniaUArRate, and are developing their final bids. Until the tender process is closed, we cannot provide any details of the technology choices of the potential contractors.

A successful meeting between the Collaboration and the Kinder Morgan Company team took place in Cortez, Colorado, on March~5, 2019, at which time Kinder Morgan's commitment to the project was reconfirmed.  The current plan is to install and commission the plant between the end of~2020 and the fourth quarter of the~2021 calendar year, allowing for extraction of the \UraniaTotalDSkProduction\ of \UAr\ by the middle of the~2022 calendar year.  The required preparation work for the installation of the plant is well understood, and the extraction site is ready for work to begin as soon as the local approvals are made for the land development permit and the remaining required funding is secured.  


%---
\subsection{Final Argon Purification: \Aria}
\label{sec:Aria}

Purification of \UAr\ will be carried out with the \AriaSeruciHeight\ tall \SeruciOne\ cryogenic distillation column, which is composed of a bottom reboiler module, a top condenser module, and \AriaCentralModulesNumber\ central modules.  
All modules have been built, certified leak-free following tests at \CERN, and received at the ``Monte Sinni'' mine of Carbosulcis in Sardinia.  
The last authorization required for the installation in the \Seruci\ mine shaft was received in April~2019.  
Installation of the platforms necessary for supporting the column starts in June~2019.  
The first batch of \UAr\ will arrive in Sardinia during the beginning of 2022.

Construction of the \AriaNuraxiHeight\ \SeruciZero\ pilot plant, which consists of a bottom reboiler, a top condenser, and a single central module, at the ``Laveria'' above ground site in Nuraxi Figus is complete.  
A leak test of the three modules was successfully completed in May~2019.
\SeruciZero\ will start operations in June~2019.
This prototype column will test all of the components of the full \SeruciOne\ column.
The \SeruciOne\ plant is estimated to be able to process \UAr\ at a rate of \AriaSeruciOneRate, obtaining a \ce{^39Ar} depletion factor of \AriaDepletionPerPass\ per pass and enabling further suppression of \ce{^39Ar} by two or three orders of magnitude. 
This will play a crucial role in the science reach of \DSls. 
Commissioning of \SeruciZero\ will begin in June~2019 following completion of the pressure test and final plant certification of compliance with the European Directive on Pressure devices (PED).

%---
\subsection{Argon radioactivity assessment: \DArT}
\label{sec:DArT}

Existing infrastructure from the \ArDM\ experiment at \LSC, Spain, will be reused to host \DArT, a detector that will measure the purity of underground argon.  \DArT\ will consist of a small \DArTVolume\ scintillation detector equipped with \SiPMs\ and inserted at the core of the \ArDM\ detector, which will serve as a veto detector.  \DArT\ will be able to test small batches of argon with a sensitivity to \ce{^39Ar} better than 1 part in \DArTUArSensitivityUltimate.  The \DArT\ Technical Design Report has been completed, approved by the \GADMC\ Collaboration, and submitted to \LSC.  Fabrication of the detector components is under way and will be completed by July~2019.

%---
\subsection{Offsite Neutron Calibrations: \ReD}
\label{sec:ReD}

The \ReD\ \LArTPC\ was developed to continue the kind of precision neutron calibrations successfully carried out by \SCENE~\cite{Alexander:2013ke,Cao:2014tw,Cao:2015ks} and \ARIS~\cite{Agnes:2018cn}. The characterization of the \ReD\ light yield and \SOne\ and \STwo\ response is now complete, and we are planning to request beam time for the \ReD\ \LArTPC\ test at \LNS\ following the completion of a neutron beam characterization measurement that will be completed by the end of May~2019.