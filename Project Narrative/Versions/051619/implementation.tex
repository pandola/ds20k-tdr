%---
\section{Implementation Plan}
\label{sec:implementation}

%{Discuss the management and technical activities that will be accomplished to prepare, initiate, execute and conclude the project.  This section should include a summary of the management plan including a description of technical readiness and project management, and an organization chart or list of key personnel and their roles (see Supplementary Documents).}

%\cmt{EP}{I feel we should exchange \DS\ with \GADMC\ almost everywhere in this section but can not take that executive decision myself} 
% \cmt{EP}{Walter had a title of Project Scientist and not Project Leader so I changed that} 
% \cmt{EP}{I changed L2 into L1 when adapting the text from the yellow book some confusion about what is called level 1 and level 2}

The tasks and technical activities for the implementation of \DSks\ have been distributed among the \GADMC\ collaborating institutions and organized into three sub-projects: the \DSks\ detector, \Urania, and \Aria. The list of specific tasks that must be accomplished for the timely completion of the detector is detailed in the work breakdown structure (WBS).  The organization of the \GADMC\ management and the list of Level 1 WBS items are summarized in Tab.~\ref{tab:WBSsummary}. A detailed schematic of the \GADMC\ management structure can be found in the accompanying Project Execution Plan.

The collaborating U.S. institutions will play critical roles in the construction and commissioning of the \DSks\ detector and the Urania facility, with a focus on the Urania plant installation and commissioning, photoelectronics development and fabrication, the cryogenics and gas handling system design and fabrication, the inner detector design and component fabrication, and the development of calibration sources and systems. A summary or the individual contributions of each institution follows.

{\bf University of California Davis:} The UC Davis group is responsible for delivering the high voltage systems, field cages, and reflector cages for the \DSps\ and \DSks\ TPCs. They will oversee the fabrication and testing of parts for these subsystems, manage their integration into the TPC, and participate in the assembly and commissioning of the detector.

{\bf University of California Los Angeles:} The UCLA group leads the \DSks\ TPC and cryogenic task. In this role, they will coordinate  the design, fabrication, and testing of the \DSps\ and \DSks\ TPC and cryogenic system at CERN and the installation of the \DSks\ systems as LNGS. They will design many of the cryogenic and gas-handling subsystems, including the argon condenser and the argon purification loop.

{\bf Fort Lewis College:} Fort Lewis will assist with the installation and commissioning of the Urania facility on-site in Colorado. They will also spearhead the revitalization of the Princeton-Gran Sasso Summer School of Physics.

{\bf University of Hawaii:} The Hawaii group leads the calibrations task. Within this role they will develop a liquid argon camera system for detector monitoring and procure, build, and upgrade the radioactive sources required for detector calibration. This effort includes the design and fabrication of systems for deploying these sources within \DSps\ and \DSks.

{\bf University of Houston:} UH is responsible for the technical coordination of the Urania project, including the preparation and installation of infrastructure at the plant site and the receipt, installation, and commissioning of the plant. UH is also designing and fabricating the support frames and wire grids for the \DSps\ and \DSks\ TPCs.

{\bf University of Massachusetts, Amherst:} UMass will lead the integration of the photo-detector modules with the TPC assembly and oversee the development of SiPMs at LFoundry with integrated through silicon vias (TSVs). They are also responsible for ensuring the Urania plant is fully integrated with the Kinder Morgan facility prior to the start of underground argon extraction.

{\bf Princeton University:} The Princeton PI is the spokesperson for \DS\ and as such will oversee the \DSks\ experiment, coordinating the resources of the international scientific and industrial team. Princeton is also responsible for the procurement of several large components of the gas-handling and cryogenics system, the continued development of low-background, SIPM-based photodetector tiles, and the on-site construction of the \DSks\ experiment.



The \GADMC\ is organized so that global directives, approvals, and monitoring are handled by a central body, which ensures that the physics performance goals are met, milestones are completed within the determined time schedule, systems are integrated, and hardware and software is of uniform quality between the projects. The Institutional Board (IB) decides on all appointments and terminations of the \DS\ management roles and approves of any major design changes of the experiment. Each institution within the \GADMC\ is represented on the IB and decisions are taken by consensus or vote. The Resources Review Board (RRB) is responsible for the overall resource planning, ensuring that resource needs are consistent with the various local and national funding sources.

Whenever appropriate, decision making is handled at the sub-project level. Project management is delegated to a Project Leader (PL) and to a Technical Coordinator (TC) for each of the three sub-projects; \DSks, \Urania, and \Aria. The PL is the scientist responsible for the project direction and for coordinating L1 sub-projects. He or she ensures that the design and construction of the project is carried out on schedule, within the cost ceiling, and in a way that meets the performance and reliability requirements determined within the framework of \GADMC\ resource planning. The TC is responsible for the project construction and the technical integration of all its components. The TC ensures the implementation of engineering standards and procedures, monitors the overall construction of detectors and infrastructure, and is responsible for the sub-projects' integration and safety.

The Technical Board (TB) is the main body for directing the execution of the \GADMC\ projects and for direct communication between the \GADMC\ management and the sub-projects. The TB is composed of Management (SP, Deputy(ies) SP, IB Chair, RC, PL's, TC's), all L1 sub-project coordinators, the Group Leader In Matters Of Safety (GLIMOS), and the environmental contact for the experiment (RAE).

\begin{table}
%\rowcolors{3}{gray!35}{}
\begin{center}
\caption[WBS summary]{Organization chart according to the Work Breakdown Structure at Level 1.}
\label{tab:WBSsummary}
\resizebox{\textwidth}{!}{
\begin{tabular}{llccc}
{\bf WBS} 			   	& Activity 									& Key Personnel 						& Key Personnel Role 			& Lead Institutions (alphabetic)\\
\hline
\multirow{3}{*} {} 		& \multirow{3}{*}{\textbf{\GADMC}} 				& C. Galbiati (Princeton)   					& Spokesperson   				& \multirow{3}{*}{All}\\
           				&                                      						& G. Fiorillo (Napoli)        					& Deputy Spokesperson  			& \\
           				&                                     						& G. Batignani (Pisa)      					& Collaboration Board Chair 		& \\
\hline
\multirow{1}{*} {\textbf{1}} 	& \multirow{1}{*}{\textbf{\DSks}} 				& A. Ianni (Princeton)  					& Technical Coordinator   			& \multirow{1}{*}{All}\\

\hline
\multirow{4}{*} {1.01} 	& \multirow{4}{*} {Photo Electronics}  			& \multirow{4}{*}{E. Scapparone (Bologna)} 	& \multirow{4}{*} {Level 1 Manager} 	& BNL, Bologna, Cagliari, \\
                                		&                                                 	     			&                                                                    	&                                                       	& LNGS, Milano, Napoli, \\
                                		&                                                        				&                                                                   	&                                                        	& Pisa, Princeton, TIFPA, \\
                                		&                                                        				&                                                                   	&                                                        	& Torino, UMass \\
\hline
\multirow{3}{*} {1.02} 	& \multirow{3}{*} {Inner Detector \& Cryogenics} 	& \multirow{3}{*} {H. Wang (UCLA)}              	& \multirow{3}{*} {Level 1 Manager} 	& Alberta, BNL, Carleton,  \\
                               		&                                                 				&                                                    			&                                                    	& CIEMAT, Davis, FNAL, \\
                                		&                                                 				&                                                    			&                                                       	& Napoli,  Roma 1,  UCLA\\
\hline
1.03                        		& Materials                   						& R. Santorelli (CIEMAT)     				& Level 1 Manager 				& CIEMAT, Krakow, Princeton\\
\hline
\multirow{3}{*} {1.04} 	& \multirow{3}{*} {Calibrations}      				& \multirow{3}{*} {J. Maricic (Hawaii)}              	& \multirow{3}{*} {Level 1 Manager} 	& BNRU, Hawaii, Krakow,  \\
                                		&                                                 				&                                                    			&                                                       	& MSU, Princeton, Temple\\
                               	 	&                                                 				&                                                    			&                                                       	& Virginia Tech \\
\hline
\multirow{2}{*} {1.05} 	& \multirow{2}{*} {Outer Detector}  				& \multirow{2}{*} {G. Testera (Genova)}       	& \multirow{2}{*} {Level 1 Manager} 	& Alberta, Carleton, CIEMAT,  \\
                                		&                                                 				&                                                    			&                                                        	& Genova, IHEP\\
\hline
1.06                        		& Electronics                   						& M. Rescigno (Roma 1)    				& Level 1 Manager 				& Roma 1, TRIUMF, Virginia Tech\\
\hline
1.07                       		& Offline                         						& D. Franco (APC)           					& Level 1 Manager 				& APC, Roma 1, Roma 3\\
\hline
\multirow{4}{*} {1.08} 	& \multirow{4}{*} {ReD}  						& \multirow{4}{*}{L. Pandola (LNS)}  			& \multirow{4}{*} {Level 1 Manager} 	& BNL, Bologna, Cagliari, \\
                                		&                                                 	      			&                                                                    	&                             				& Genova, LNGS, LNS, \\
                                		&                                                       			 	&                                                                   	&                              				& Napoli, Pisa, Roma 1, \\
                                		&                                                        				&                                                                   	&                              				& APC, UCLA, Roma 3 \\
\hline
\multirow{2}{*} {1.09} 	& \multirow{2}{*} {Prototype}  					& \multirow{2}{*} {G. Fiorillo (Napoli)}       		& \multirow{2}{*} {Level 1 Manager} 	& UCLA, Napoli, Princeton,  \\
                                		&                                                 				&                                                   			&                                                      	& Roma 1\\
\hline
1.10                      		& Outer Cryostat              						& M. Nessi (CERN)           					& Level 1 Manager 				& CERN, UCLA, INFN, LNGS\\
\hline
\multirow{2}{*} {\textbf{2}} 	& \multirow{2}{*}{\textbf{\Urania}} 				& A.Renshaw (Houston)  					& Technical Coordinator   			& BNL, Carleton, Houston, \\
           				&                                         					& M. Simeone (Napoli) 					& Project Leader  				& Napoli, PNNL\\
\hline
2.01                         		& Project Manager Oversight	                            	& B. Walsh (BNL)		        				& Level 1 Manager  				& BNL \\
\hline
2.02                         		& Plant Fabrication and Delivery                            	& M. Simeone (Napoli)        				& Level 1 Manager  				& Carleton, Napoli, PNNL \\
\hline
\multirow{2}{*} {2.03}		& \multirow{2}{*} {Site Prep. \& Installation} 		& \multirow{2}{*} {A.Renshaw (Houston)}       	& \multirow{2}{*} {Level 1 Manager} 	& Carleton, Houston, PNNL,  \\
                                		&                                                 				&                                                    			&                                                       	&  Princeton, UMass, Temple \\
\hline
\multirow{3}{*} {2.04}		& \multirow{3}{*} {Extraction}  					& \multirow{3}{*} {H. Back (PNNL) }      		& \multirow{3}{*} {Level 1 Manager} 	& BNL, Canada, Carleton,  \\
                                		&                                                 				&                                                    			&                                                       	& Fort Lewis, Houston, PNNL, \\
                                		&                                                 				&                                                    			&                                                        	& Princeton \\
\hline
2.05                         		& Storage and Shipping		                            	& M. Boulay (Carleton)	        				& Level 1 Manager  				& Carleton \\
\hline
2.06                        		& Shutdown				                            	& A.Renshaw (Houston)        				& Level 1 Manager  				& Houston \\
\hline
\multirow{2}{*} {\textbf{3}} 	& \multirow{2}{*}{\textbf{\Aria}}  					& R. Tartaglia (LNGS)  					& Technical Coordinator  			& Cagliari, LNGS, Princeton, \\
           				&                                         					& L. Mapelli (Princeton) 					& Project Leader  				& Napoli \\  
\hline
\multirow{2}{*} {3.01} 	& \multirow{2}{*}{Seruci-0 and Seruci-I}			&  \multirow{2}{*} {F. Gabriele (LNGS) }  		& \multirow{2}{*} {Level 1 Manager}  	& Cagliari, Fermilab, LNGS, \\
                                		&                                                					&                                                    			&                                                        	& Napoli, Virginia Tech \\
\hline
\multirow{2}{*} {3.02} 	& \multirow{2}{*} {DArT in ArDM}				& \multirow{2}{*} {W. Bonivento (Cagliari)}       	& \multirow{2}{*} {Level 1 Manager} 	& Cagliari, CIEMAT, LNGS,  \\
                                		&                                                 				&                                                    			&                                                     	&  ETHZ\\
\hline  
\textbf{4}                           	& \textbf{Infrastructures}                  				& A. Ianni (Princeton)        					& Level 1 Manager  				& Carleton, LNGS, Princeton\\    
\hline  
\multirow{2}{*} {\textbf{5}} 	& \multirow{2}{*} {\textbf{\NOA} }  				& \multirow{2}{*} { E. Scapparone (Bologna)}    	& \multirow{2}{*} {Level 1 Manager} 	& Bologna, LNGS, Princeton  \\
                                		&                                                 				&                                                    			&                          				&  Temple\\
%\cellcolor{white}{\bf ....}
%								&\cellcolor{white}\\
\end{tabular}}
\end{center}
\end{table}