%---
\section{\DSk\ inner detector}
\label{sec:Inner-Cryogenics}

\subsection{TPC}
The conceptual design of \DSk\ is being finalized with acrylic as underground argon vessel surrounded with regular argon which act as veto. \\
%In collaboration with the IHEP group in Beijing, we secured the acrylic needed for the \DSk\ TPC vessel from the same acrylic of the JUNO project produced by DonChamp inc in Changzhou China. \\
 The design concept of the TPC will be verified with a staged effort using a small prototype at CERN and the TPC parts are being fabricated by members of the collaboration. The small prototype will be tested at CERN using the CERN existing cryogenics system.  A 1-ton prototype with full features of the \DSk\ TPC has been designed with a cryostat already fabricated and is currently located at CERN. The prototype TPC will be fabricated with low radioactive materials that will enable its potential use as a dedicated detector for low mass WIMP search at LNGS.  An engineering team from the \DSk\ collaboration is currently actively working at CERN together with the CERN neutrino platform engineering team to finalizing the design requirement of the TPC for the protoDUNE style cryostat. \\
The major change in the \DSk\ TPC design is the field shaping electrode construction. Instead of low background copper ring, we are planning to use a conductive transparent polymer, Clevios, to form both the field cage ring and the cathode and anode. This work is actively being worked on at CERN.

\subsection{Cryogenics}

Pending a safety review for both CERN and LNGS requirements, two cryogenics systems will be built for testing: a  full scale cryogenics system for DarkSide-20k, and a dedicated system for the 1-ton prototype TPC. The full cryogenics system test is expected to be completed mid 2019 and the cryogenics system for the prototype will be completed later in 2019.









