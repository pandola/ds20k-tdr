\section{Material assays}
\label{sec:Materials}


%({\it roberto.santorelli@ciemat.es})  \\

The radioactive contamination of 17 samples has been measured during the first months of 2019, with 20 assays in total. 60 results have been obtained concerning the contamination of the upper, middle and lower $^{238}$U and $^{232}$Th chains. The material database is now fully operational and includes all the  results of the DarkSide assay campaign.  The radioactive budget has been refined including the latest Monte Carlo results, allowing a detailed evaluation of the assay results. None of the measured samples has been identified as a showstopper. The analysis  of acrylic sheets directly provided by the supplier  is ongoing. This is a critical measurement given the large amounts of this material both in the TPC and in the veto.
In parallel with the  calculation of the background produced by the  bulk contamination, the working group is now addressing the background given by the  activation  of the materials due to cosmic rays. Preliminary results have been produced for copper, stainless steel and argon and are now being evaluated. Finally, protocols for material protection and transportation are currently being defined in order to reduce the surface contamination given by Rn daughters plate-out  and Rn diffusion inside the bulk.

