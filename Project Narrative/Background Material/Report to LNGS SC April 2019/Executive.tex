%---
\section{Collaboration matters}

We highlight three important developments that took place since the last Scientific Committee meeting.\\
We have been informed that the Italian Ministry for Education, University and Research (MIUR) approved a grant of \euro18.4M in response to the INFN-LNGS proposal submitted to the ``Programma Obiettivo Nazionale'' (PON), see\\ \small\url{http://www.ponricerca.gov.it/media/395531/m_piaoodpfsrregistro-decreti-r-000046114-03-2019.pdf}.  The funding request included capital funding for \DSk\ infrastructure for \euro15M.\\
Following contacts between the INFN and IHEP, China, leadership, an agreement was reached to produce the acrylic material for both the TPC and the Veto detectors in China.  The production will be carried out by the company DonChamp inc in Changzhou, that is providing the acrylic for the JUNO experiment.  IHEP collaborators already requested the necessary capital funding to the Chinese Ministry of Science for purchasing the acrylic for the VETO.\\
Following contacts between the INFN and the US Department of Energy (DOE) leaderships, detailed discussions are ongoing to establish a possible capital funding contribution from DOE to the \DSk\ experiment, through Brookhaven National Laboratory (BNL) and Fermilab (FNAL).\\
Three new groups were admitted officially in the DarkSide Collaboration.\\
The Brookhaven National Laboratory (BNL) group has assumed responsibilities in the areas of photoelectronics, TPC, and offline calibrations.  The group at {\it Museo Storico della Fisica e Centro Studi e Ricerche Enrico Fermi}, has assumed responsibilities on Construction Database, photoelectronics and outreach.  The group at Laboratori Nazionali di Legnaro (LNL) has assumed responsibilities on precision cleaning of copper and  on the establishment of cleaning protocols and in data analysis.\\
At present, the DarkSide collaboration is composed of 59 institutions and 371 scientist from 14 nations: Brasil, Canada, China, France, Greece, Italy, Mexico, Poland, Romania, Russia, Spain, Switzerland, United Kingdom, United States of America.\\
The Collaboration voted to choose \Argo\ as the name of the future, ultimate detector with a mass in the range of the multi-hundred tonnes.\\
The document entitled {\it Future Dark Matter Searches with Low-Radioactivity Argon} was submitted as {\em Input to the European Particle Physics Strategy Update 2018-2020}, and can be found at the link \url{https://indico.cern.ch/event/765096/contributions/3295671/}.