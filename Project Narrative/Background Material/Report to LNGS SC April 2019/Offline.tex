%---
\section{Science and simulation}
\label{sec:Offline}

%---

\DSk\ Monte Carlo code is in an advanced status: the Geant4-based package (g4ds) is currently completed, and undergoes  frequent updates following the changes of the detector design. Recently, we have also implemented  a preliminary version of the electronics simulation, which has allowed the development of the event reconstruction code. ReD data are used to test reconstruction. Simulations are widely employed for different aspects of the experiment (e.g. background budget, optical response, neutron veto optimization), witnessing the advanced status of the code. \\
From the "Science" side, we are very active in the determination of the sensitivity of \DSk\ and ARGO to Supernova neutrinos exploiting the flavor insensitive coherent scattering channel.  
Such detection has a unique potential in providing information on total energy emitted by the supernova neutrino when compared with results from large scale detectors like DUNE, HyperKamiokande, and JUNO, which are mostly sensitive to electron neutrinos.
At the same time, we are still investigating potential in extending the low-mass WIMP limit with different configurations of the \DSk\ prototype and vetoes.