\section{\DSf}
\label{sec:DSf}

The \DSf\ high-mass WIMP search using the blind analysis of \DSfDdLTPostQualCut\ of data has been published in Physical Review~D.  It was surpassed as the most sensitive WIMP limit in Argon for masses above \SI{70}{\GeV\per\square\c} (by our DEAP colleagues, that are now \DSk\ collaborators) only in February of this year.\\
World-leading limits from \DSf\ were also published in Physical Review Letters, for interactions of  low-mass WIMPs and for dark matter particles coupling primarily to electrons.\\
\DSf\ operations continue, primarily to diagnose problems that occurred during the February 2018 blackout.  The experiment is functional, but the data quality is marginal due to contamination whose nature and origin is still not understood.  While the impact of this contamination has been mitigated with continued purification, the rate of improvement is extremely slow, and, after a year, the data are still not of physics quality.\\
   Currently, the limited resources and manpower available to the project are focused on the commissioning of the system that would allow permanent, safe storage of the full inventory of underground argon currently deployed in \DSf.
