%---
\section{Veto}

The design of the  veto is based on the concepts already  reported into the Report to the XLX LNGS Scientific Committee. The veto detector is composed by three volumes:
\begin{itemize}
    \item an inner volume of active liquid atmospheric argon  (Inner argon Buffer, IAB) surrounding the TPC;
    \item a passive shell of acrylic (PMMA) loaded with gadolinium with octagonal shape (GdA) mounted around the TPC. The IAB is in between the TPC and GdA. 
    The  acrylic shell loaded with Gd surrounds the TPC in all the directions (lateral, top and bottom with the exception due to the service holes). 
    \item an outer  active volume of atmospheric argon (Outer Argon Buffer, OAB) 
\end{itemize}
 A copper cage (Faraday cage) provides the optical insulation from the rest of the argon external to OAB and, at the same time, it realizes the necessary electric  shield.\\
We have performed a series of Monte Carlo studies and, based on that, we have established that
the required thickness of both the IAB and OAB is 40~cm, with no performance penalty for a thickness increase. The required thickness of GdA is 10 cm. The  mass fraction of Gd  in the acrylic should be between 1\% and 2\%. \\
The TPC and GdA are shaped as polyhedron with  octagonal cross section. The apothem of the inner face of GdA octagon
is 225 cm; the internal height of the GdA is 400 cm.
Assuming a density of 1.18  g/cm$^3$ the mass of the acrylic loaded with gadolinium  is 11.7 tons. \\
Neutrons   are moderated by collisions (mostly with  hydrogen)  in the acrylic. The presence of Gd  ensures the emission of multiple high energy  $\gamma$-rays after the neutron capture. With Gd concentration between 1 and 2\% by weight, capture of neutrons  in Gd happens with  about 54\% probability and in argon with 24\%. The remaining neutrons are caught in hydrogen with 16\% probability and copper (8\%). 
 Note that the GdA acts as moderator and neutron capture agent and then there are no requests about its transparency to the scintillation light.
 $\gamma$-rays following the n capture interact in the IAB and OAB producing scintillation light that is detected by light sensors mounted on the two sides of GdA and facing both the IAB and OAB. \\  
The IAB and OAB are segmented  into vertical sectors using thin acrylic panels. The sectorization has the purpose of  reducing the pile-up event rate due to the decay of $^{39}$Ar  and to obtain a sufficiently high  photoelectron yield.\\
The precise number of sectors is going to be optimized; as a reference we assume to have 5 sectors along each edge of the octagon, both in the IAB and in the OAB volumes. \\
A sandwich made of a proper 3M foil (reflector) attached to a thin acrylic sheet on which we make  TPB coating on the face opposite to the one with 3M foil will be built. This will be attached to the
copper cage and to the external wall of the TPC. The panels of the sectors will be realized by 2 sandwiches coupled together.\\
%The  presence of the reflectors trap the Argon light inside the volume defined by the vertical sectors,enhances the  light collection and consequently the veto energy resolution, crucial to reject $^{39}Ar$ signals. \\
Sectors are not liquid tight and the proper argon  flow should be ensured both during filling and re-circulation.\\
The argon light is detected by SiPM tiles (2000 in the IAB and 1000 in the OAB) with the same size as the one of the TPC. Montecarlo simulations show that the expected light yield is about 2 photoelectrons/keV.\\
The same company that will produce the acrylic panels is available to make
small scale laboratory tests to mix a proper Gd compound with the acrylic and then produce the  necessary amount for DarkSide. As we do not require  the doped plastic to be transparent, several difficulties faced with metal loading of organic materials to make scintillators are reduced. However the selection of the proper Gd compound requires some R\&D work that will start during the next weeks. Assays of the acrylic produced by this company performed by another collaboration shows that the the U and Th
contamination should be acceptable for the VETO (ppt level). 
Additional assays will be performed by \DSk\ together with the assay of Gd compounds and of the final doped acrylic.
