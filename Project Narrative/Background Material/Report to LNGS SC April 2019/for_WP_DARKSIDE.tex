%---
\section{Executive Summary}


In early 2018 the \DS\ Collaboration  reached the milestone of its \DSfs\ program by publishing results from a \DSfDdLTPostQualCut\ campaign with a two-phase \LAr\ time projection chamber (\LArTPC) in operation since 2013 in the underground Laboratori Nazionali del Gran Sasso (\LNGS)~\cite{Agnes:2018vl,Agnes:2018fg,Agnes:2018ft}.

The outcome of the high-mass WIMP dark matter search is a null result (see~\reffig{DSfResults}), delivering on the promise of zero-background and producing the best limit with an argon target (see~\reffig{DSklSensitivity}).

The extremely low background, high stability, and low \DSfTriggerSTwoAnalysisThreshold\ analysis threshold of \DSfs, enabled a study of very-low energy events, characterized by the presence of the sole ionization signal (see~\reffig{DSfResults}), which resulted in the world-best limit for low-mass dark matter searches in the mass range \DSfLowMassBestRecordRange~\cite{Agnes:2018fg} (see~\reffig{DSklSensitivity}).  The same analysis stream also produced very competitive limits for the scattering of dark matter into electrons~\cite{Agnes:2018ft}.



The first step of the \GADMC\ (the Global Argon Dark Matter Collaboration) is the \DS\ program at \LNGS.  \DSk\ (\DSks) is the search for high-mass dark matter, approved by \INFN, \NSF, and \LNGS\ in 2017.  Based on the recent success in low-mass searches, the Collaboration has decided that it will propose to \LNGS\ the construction and operation of \DSl\ (\DSls), a search specialized for discovery of dark matter in the low mass region.  \DSls\ will exploit elements of the \DSp\ (\DSps) prototype detector currently in construction at \CERN.


\begin{figure*}[!t]
\includegraphics[width=\textwidth]{Figures/CombinedExclusionLimits_nufloor.pdf}
\caption[Current \DM\ limits and sensitivities for future experiments.]{Current \DM\ limits and sensitivities of past and  future experiments. The \DSls\ curves are explained in section\ref{sec:DSl}}
\label{fig:DSklSensitivity}
\end{figure*}