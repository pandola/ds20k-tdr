%---
\section{Argon procurement and purification}
\label{sec:Ar}


%---
\subsection{Underground Argon Extraction and Purification: \Urania}
\label{sec:Ar-Urania}

%({\it simeone@unina.it})  \\
The Urania project has made significant strides in the last months, most importantly with the opening of the tender for the construction of the argon extraction plant by the \INFN.  The opening of the tender has officially marked the start date of the project, and the timeline for the extraction of the \UraniaTotalDSkProduction\ of \UAr\ required for the \DSks\ experiment has now been set and has been integrated with the overall schedule for \DSks.  It is now expected that the tender process will close at the beginning of the fourth quarter in the 2019 calendar year, and at that point a contract will be signed for the construction and delivery of the plant to the extraction site in Colorado, USA.  \\
The Urania project team had an on-site meeting with the Kinder Morgan team on March 5, 2019.  The meeting was a re-kick off of the project work that will take place at the Kinder Morgan Doe Canyon facility, in preparation for the installation of the plant that will take place towards the fourth quarter of the 2020 calendar year.  This meeting was a huge success in getting the point of contacts established between the two parties, and action items have been assigned which will be covered by the end of the 2019 summer.  The current plan is to install and commission the plant between the end of 2020 and the fourth quarter of the 2021 calendar year, allowing for extraction of the \UraniaTotalDSkProduction\ of \UAr\ by the middle of the 2022 calendar year.


\subsection{Final Argon Purification: Aria}
\label{sec:Ar-Aria}

Seruci-0 is the pilot plant almost  completely installed  in one of the outdoor assembly hall at Nuraxi Figus, Italy\\
In the 2018 there were some meetings with the fire brigades and with local authorities in order to clarify some details and to continue in the authorization process.
Offices and storehouse are completed and operational.
Close to the storehouse a protected area has been reserved to storing and using of gas-bottles need for the operations.\\
All the column components have followed the agreed procedure: realized and tested at Polaris, transferred and tested at CERN, delivered and  tested at CarboSulcis, and installed.
All the Seruci-0 components and most of the accessory plants reached Sardinia in Spring 2018. They have been stored in the warehouse before assembly. In Summer 2018 the three main elements [Bottom, Module1 (Middle) and Top] have been assembled together in the devoted structure already built and put in place.
In the following months all leak-check tests have been performed in order to guarantee the foreseen and designed tightness to the whole apparatus. Moreover, most of the plants have been installed and assembled in the area surrounding the column itself.
To date, two concrete platform have been built: they will host the liquid nitrogen dewar and the  accessory plants, the cooling machine (chiller) and a box needed to host people on duty-shift and to organize the slow control.
The above cited components will be assembled in the months of March-April.\\
Concerning Seruci-1, the complete column plant, 350m tall to be installed in one of the well at Nuraxi-Figus, all the documents  needed for the authorization request have been sent in May 2018 to the competent authorities. In the last months several meetings have been held, both with Fire Brigades and with other local and county offices and entities.\\
In 2018 a complete cleaning and preparation of the well was  performed.
A well defined coring procedure has been ended by January 2018. After the examination of the rocks samples, the design for the Seruci-1 supporting structure has been detailed. The tender has been completed in Autumn 2018: at the beginning of 2019, a "carbon steel sample platform" has been delivered to Seruci and installed in the well: the test has been successful; it has been very useful in order to define all the needed details. The goal is to receive all the platforms by the end of April 2019 and to install them inside the devoted them by the end of 2019.\\
An additional storehouse has been selected in order to provide "confined area" for storing and operation at Seruci site. It has been completely refurbished and equipped with TV-CC system for security reason.\\
The elements composing Seruci-0 and all the modules for Seruci-1 have been successfully tested at CERN and delivered to CarboSulcis by the end of 2018. To date, 15 modules have been "parked" in the Seruci-0 warehouse and 12 of them in the Seruci-1 warehouse. \\
As far as funding is concerned, in 2017 the first (RAS1) agreement that involves INFN and Regione Autonoma della Sardegna (RAS) has been officially signed. The agreement is focused on the relationship between the Sardinia Region and the INFN and it deals with the realization of the Seruci-plant in the "Iglesiente County" and with the possible industrial and commercial spin-off that the research project Aria might create in the future in different fields of applied technology. In 2018, two additional agreements (RAS2 and RAS3) have been signed.\\
In 2018 several meetings of the so-called "comitato d'indirizzo" (Steering Committee) have been held.




%({\it roberto.tartaglia@lngs.infn.it})  \\







\subsection{Argon radioactivity assessment: DArT}
\label{sec:Ar-DArT}

An Expression of Interest for the experiment, re-using the ArDM detector at LSC, Spain, to host a small 1 litre detector containing the argon to be tested, with sensitivity to $^{39}$Ar depletions larger than 10,000 compared to atmospheric argon,   was submitted and presented at the Scientific Committee of the LSC on December 12th. The Committee recommended to proceed to a Technical Design Report.  The document is ready to be submitted to the  laboratory in Spain and is in the latest approval phases by the \DSk\ Collaboration.\\
In the meantime hardware activities have already started and the preparation of the experiment is in full steam.

%({\it walter.bonivento@ca.infn.it})  \\



