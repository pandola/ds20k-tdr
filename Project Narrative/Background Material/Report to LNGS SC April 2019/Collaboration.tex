%---
\section{The  \DSl\ experiment}
\label{sec:DSl}

In a complete synergy with \DSks, \DSls\ will utilize the \DSp\ detector in construction at \CERN\ to support tests of the \DSks\ elements, and in particular of the \DSkPdms~\cite{DIncecco:2018fx,DIncecco:2018hy}.  Thanks to the use of the low-background \SiPM-based \DSkPdms, of a low-background cryostat, and of an ultra-low background argon target purified by the \Aria\ cryogenic distillation column, \DSls\ will perform a competitive and compelling exploration of the low-mass discovery region, reaching into the neutrino floor.

\reffig{DSklSensitivity} summarizes the physics potential of \DSls.  The leading backgrounds determining the sensitivity will be the residual contamination in \ce{^39Ar} of the target and the radioactive contamination of the \DSkPdms, though in the curves shown the background from solar neutrinos is also properly taken into account, as it impacts the curves with the largest sensitivity. The first projected curve shows 
 %corresponding to contaminations of \SI{700}{\micro\becquerel\per\kg} and \SI{2000}{\micro\becquerel\per\pdm}, i
  what is {\em immediately} attainable, 
%  by building a \DSpApproximateMass\ customized \LArTPC\ 
  operating the \DSp\ underground 
  within an efficient veto, 
  %and operating it 
  with a \UAr\ fill with purity corresponding to that achieved in \DSf\ (\DSfUArArThreeNineActivity) and with the radio-purity currently achieved for the \DSkPdms\ (\SI{\sim2}{m\becquerel\per\pdm}), with an exposure of 80,000  \si{\kg d}.
 It is possible and practical to attain the sensitivity level of the second and third projection curves, 
 %corresponding to the level of \SI{100}{\micro\becquerel} and \SI{10}{\micro\becquerel} in the same units, 
 by progressive, order of magnitude additional suppression of the \ce{^39Ar} background (which can be achieved by cryogenic distillation with \Aria\ and/or by suppressing atmospheric leaks in the underground argon extraction). To improve further the sensitivity it is mandatory to reduce the 
  background induced by the \DSkPdms\ down to the equivalent  \SI{\sim0.1}{m\becquerel\per\pdm}
(which can be achieved by the use of ultra-pure substrates, such as fused silica or silicon, and/or by shielding the \DSkPdms\ with ultra-pure light guides that are under study), as shown by the fourth curve. For the latter case a new low background copper cryostat of   \SI{\sim0.01}{m\becquerel\per\kg} will need to be built to match the background reduction from the PDMs. In view of the potential improvement of the  sensitivity, we anticipate that   further reduction factors in the equivalent PDM radio-activity are going to be the subject of targeted R\&D. 
 %, made of acrylic, fused silica).  
 %Owing to the interference of nuclear recoils induced by solar neutrinos, no further significant gains in sensitivity would achieved by pushing the purity of \ce{^39Ar} and of \DSkPdms\ in the range of \SI{1}{\micro\becquerel}.

{\it It is thus possible and practical to develop at \LNGS, building on the success of \DSf, a world-leading program able to push the sensitivity for low-mass dark matter searches down to the neutrino floor.}

The world-leading low-mass results of \DSfs\ were enabled by the study of the (sole) ionization signal from very low energy events.  The analysis threshold is of \DSfLowMassThresholdEe\ (\DSfLowMassThresholdNr), corresponding to \DSfLowMassThresholdE\ extracted from the liquid target, with each electron producing in average, by electroluminescence, \DSfLowMassThresholdPE.  Since above  7$e^-$  of threshold the residual background in \DSfs\ is completely characterized and accounted for by known sources, the \Geant-based \GFDS\ Monte Carlo package~\cite{Agnes:2017cz} permits to project the sensitivity of future experiments precisely and accurately.  

For the sensitivity curves of \DSls\ the  threshold of 7$e^-$ was assumed, except for the last one for which a threshold of 4$e^-$ was assumed.

The \DSps\ \TPC\ is a scaled-down version of that of \DSks, will be operated in a low-radioactivity stainless steel cryostat.  The total height of the \TPC\ active volume is \DSpTPCHeight\ and the diameter of the circle inscribed in the octagon surface covered by PDMs is \DSpTPCDiameter.  The total (active) \UAr\ target mass of \DSpTotalMass\ (\DSpActiveMass) matches perfectly the anticipated \AriaSeruciOneRate\ rate of isotopic separation of \ce{^39Ar} from \ce{^40Ar} of the Aria column, which would enable the operation with a strong suppression of \ce{^39Ar}, necessary for the ultimate scientific objective of \DSls.  The baseline option for getting the UAr for  \DSl\  is from Urania. However, if the \SeruciOne\ plant and the detector were available in LNGS much beforehand, it is conceivable to obtain depleted argon from AAr in just one year of operation of \SeruciOne.

%The \TPC\ will be equipped with the same \SiPM-based \DSkPdms\ developed for \DSk.  The top and the bottom will consist each of \DSpSQBsNumber\ square boards (\SQBs), containing \DSkSQBPdmsNumber\ \DSkPdms, of \DSpTRBsNumber\ triangular boards (\TRBs), made of \DSkTRBPdmsNumber\ \DSkPdms, defining an overall surface in shape of an octagon.  The total number of \DSkPdms\ is \DSpPdmsNumber.  The radio-purity of the \DSkPdm\ is of paramount importance given their proximity to the fiducial volume. 
% We anticipate the need of replacing the substrate of the \DSks\ \DSkPdms\ with ultra-low background materials (silicon, fused-silica, etc.).  

Operation of \DSps\ at \LNGS\ could be made easily possible by the integration in the DarkSide facilities in the underground Hall~C of \LNGS.  There are several options and a final choice for location and for shielding and vetoing strategy is under study.  In particular, \DSps\ could be operated within a passive shield or within an active veto.

For the first likely operation within a passive veto, the basic requirements are the abatement of the environmental background due to \gr~\cite{Malczewski:2012gi} and neutron~\cite{Wulandari:2004hm} background in the underground halls of \LNGS.  A possible option would consist of surrounding the cryostat with thick layers of polyethylene and lead as successfully explored by the XENON Collaboration for XENON10~\cite{Angle:2008ki} and later by the XENON100 Collaboration~\cite{Aprile:2010jn}.  Another, more convenient option would consist of placing the detector at the center of a stack of polyethylene tanks filled with ultra-high purity water, as successfully explored by the DarkSide Collaboration for the shielding of \DSt~\cite{Alexander:2013jn}.  In case of use of a passive shielding two locations could be available: the platform built by the Darkside Collaboration to house \DSts\ in Hall~C of the underground \LNGS~\cite{Alexander:2013jn} or other areas within the confines allocated for \DSks\ in Hall~C of the underground \LNGS.

Alternatively, an active anti-coincidence veto would further reduce the backgrounds induced by \grs\ and cosmic rays.  \DSps\ could be housed within the current active shielding of \DSfs~\cite{Agnes:2016fz,Agnes:2015gu}, or be surrounded by a new plastic scintillator layer.  
%In the long term the detector siting could be within the Proto-DUNE cryostat in construction for housing \DSks\ in Hall~C of the underground \LNGS.

%Specialized triggers for \STwo-only signals will also be developed, with the goal to be fully efficient already for signals induced by \SI{1}{\el}.

%When filled with \UAr\ and within an appropriate shielding, it is expected that the \DSps\ will trigger at approximately \DSpBackgroundRate, mostly induced by \grs\ from external contaminations, producing a data stream of about \DSpDataRate.


%1) reducing the uncertainties of the NR ionization yield (hence the mean value) should (slightly?) improve also current limits (>1.8 GeV)
%2) identifying the statistics governing NR ionization fluctuations will allow lowering the threshold down to (potentially) ~0.6 GeV, with current data.  
An improved knowledge of the ionization distribution of nuclear recoils is needed to reduce the uncertainties in the expected signal yield above the analysis threshold and thus improve the sensitivity at the lowest masses.  


As part of the \DSks\ effort, we plan to study low energy nuclear recoils performing direct measurements of scintillation and ionization yield using a neutron-beam in the \ReD\ experiment, 
%with the aim of demonstrating a directional sensitivity of a \LArTPC\ for high (\SI{>50}{\keVr}) energy scatters. 
and we  plan to perform dedicated studies in the energy range of interest for low mass DM detection (\SI{<1}{\keVr}), with the specific goal of a first direct measurement of the ionization yield in liquid argon and, possibly, of establishing a realistic and detailed model for fluctuations of ionization of nuclear recoils.

%Our  sensitivity estimate for \DSls\ accounts for the $\Ne = \DSfLowMassKnownBckThresholdE$ threshold beyond which the low-energy spectrum of \DSfs\ is completely accounted for known background sources (see \reffig{DSfResults} and \refref{Agnes:2018fg}).  We account for background in the steel of the cryostat at the level of \DSlCryostatUCont\ for \ce{^238U}, of \DSlCryostatUCont\ for \ce{^232Th}, and \DSlCryostatUCont\ for \ce{^60Co}.  We assume that contamination in the \DSkPdms\ will be lowered to \DSlCryostatPDMCont\ by substitution of substrates with ultra-high purity materials.  We also assume that any \ce{^85Kr} contamination will be completely removed through the combined action of the \Urania\ and \Aria\ plants, and also assume a \DSlUArCont\ \ce{^39Ar} contamination of after isotopic separation with \Aria\ if necessary.  We apply a fiducial radial cut of \DSlRadialCut, resulting in a fiducial mass of \DSlFiducialMass.  No fiducial cut is applied in the drift direction.  We calculate \SI{90}{\percent} C.L. exclusion limits corresponding to an exposure of \DSlExposure, which can be obtained in a \DSlRunTimePlanned\ data taking, and show the very compelling results in \reffig{DSfSensitivity}.