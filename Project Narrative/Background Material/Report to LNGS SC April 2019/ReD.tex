\section{ReD} \label{sec:ReD}

%({\it pandola@lns.infn.it})  \\

After the LNS, Italy, test beam of September 2018, the ReD TPC has been transported back to Naples,  to 
complete the characterization and commissioning. Tests have been performed to characterize the 
basic TPC performance in terms of light yield, uniformity, electric field configuration and S2/S1 
ratio in double phase. The system was calibrated with ordinary $\gamma$-sources and with an 
internal $^{83m}$Kr source, which generates a uniform distribution of mono-energetic events. 
Activities are still ongoing to characterize the extraction and multiplication fields.\\
A new test beam is planned in Spring 2019 at LNS, aiming for a more detailed characterization of 
the neutron beam. This test will be performed without the TPC, but only with the Si telescope, to 
detect $^{7}$Be nucleus which accompanies the neutron, and the liquid scintillator cells as neutron 
detectors. The test is targeted to improve the layout used in the September 2018 test beam, and 
specifically the alignment of the system, the beam-correlated neutron background and the beam 
divergence. After the completion of the test beam, and if the key requirements are met, the 
TPC and the cryogenic system will be shipped back to LNS, for a re-assembly of the entire ReD 
setup. One week beam-time for a physics measurement will be likely planned in June-July. \\
The ReD project got a scientific approval by the Scientific Committee of LNS (PAC). A five-week 
beam allocation was granted for 2019.   
