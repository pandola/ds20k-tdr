%---
\section*{Project Execution Plan Outline}

\section{Introduction}

The aim of the \DSls, \DSks\ and \Argo\ experiments is to probe the existence of low- and high-mass weakly interacting massive particle (\WIMP) dark matter at unprecedented sensitivities, as well as measure solar neutrinos and neutrinos coming from supernova bursts.  Based on previous experience gained in designing, building and operating dark matter detectors, the Global Argon Dark Matter Collaboration (GADMC) is well suited to build on prior successes, as well as utilize modern technologies to further boost sensitivity.  The overall goal of the program will be to push the level of sensitivity to the so called ``neutrino floor'' for both low- and high-mass \WIMPs, make a measurement of the CNO contribution to the solar neutrino flux, and contribute to the next observation of a galactic supernova and the early warning of said event.

The search for low-mass \WIMPs\ will require the use of a tonne-scale liquid argon time projection chamber (\LArTPC), operated in an dual-phase mode with ultra-low backgrounds.  The \LAr\ target itself should be low-radioactivity argon procured from underground sources, and the photosensors should be cryogenic silicon photomultipliers (\SiPMs).  With ultra-low backgrounds and the ability to detect single drift electrons, the lowest energy threshold of any liquid-noble gas will enable \DSls\ to push the search for \WIMP\ dark matter with masses between \SI{1}{\GeV\per\c\squared} and \SI{10}{\GeV\per\c\squared}.  

The search for high-mass \WIMPs\ requires the push to larger target volumes, but similar hardware requirements.  \DSks\ will provide a \DSkFiducialMass\ dual-phase \LArTPC, also utilizing low-radioactivity argon and \SiPM-based photosensors.  In addition, \DSks\ will also utilize cryostat technology developed for the DUNE experiment, allowing for the scaling of future detectors inside a \LAr-based veto system and minimizing the amount of higher-activity components in close vicinity to the inner detector, such as stainless steel.  With the exposure planned for \DSk, somewhere between one and three neutrino events will be observed, anything beyond this would be a strong indication for the detection of dark matter.  

A detailed understanding of the scientific merit of these two detectors and the technical requirements for each will be broken down in the full project execution plan (PEP).  

The push beyond \DSk\ starts with the two detector described above, and further development of a multi-hundred tonne fiducial mass detector based around the same technology.  The exposure planned for Argo will allow unprecedented measurement of the CNO solar neutrino contribution and neutronization burst of a galactic supernova.  While the implementation of the Argo detector is not a part of this project, the development of such a facility will parallel the work that is connected to this project.  

This project offers a unique set of broader impacts that benefit society in measurable ways.  The technologies developed benefit society at large in a variety of ways. Broadly, these technologies fall into categories of medical diagnostics, advances in photon detectors, and isotope separation.  These benefits include technological advances and specific educational opportunities for underrepresented populations that can increase the diversity of a STEM-competent work force.  We believe the diversity of the partner institutions creates opportunity and enhances the overall success of the program.

%\subsection{Scientific Objectives}

%\subsection{Scientific Requirements}

%\subsection{Facility/Infrastructure}

%\subsection{Scientific and Broader Societal Impacts}


\section{Organization}

Because of the scale of the project, it has brought together researchers from more than 15 different countries, all partnering together to provide the resources required for the successful implementation of the project.  Internal governing boards and committees have been formed to manage the internal parts of the project.  A technical board has been established for the technical coordination of the project and to serve as the body for organizing the work division between those institutions involved, taking into account resources and areas of expertise.  Within the technical board there is a member who plays the role of point of contact with each of the partnering funding agencies, responsible for communicating with the funding agencies and sharing resources information with the technical board.  Each of the supporting funding agencies will have a point of contact who is a collaborating member and within the technical board, allowing for the flow of important technical and resource information between the technical board and the funding partners. 

The project has also established an outreach and community relations working group, responsible for organizing events and coming up with new ways to engage the communities local to the participating institutions and project sites.  This committee will head the coordination of educational outreach programs for students of all ages, and will better connect the local public with the project and the science that will be performed with the new facilities.  Two other bodies, the Speaker's Board and the Publications Committee, will serve to make sure project progress and results are shared with the wider scientific research community 

%\subsection{Internal Governance \& Organization and Communication}

%\subsection{External Organization and Communication}

%\subsection{Partnerships}

%\subsection{Roles and Responsibilities}

%\subsection{Community Relations and Outreach}


\section{Design and Development}

A project development plan has been established, and focuses on each of the major aspects of the overall project.  This plan includes the roadmap for finalizing all technical designs and developing the fabrication facilities that will be utilized for construction.  The design and development aspect of the project is minimal in comparison to the overall project scope, stemming from the fact that the project is ready for implementation after the 4 years of design and development.  The plan also includes a clear schedule for meeting the project goals within the overall project schedule.  

%\subsection{Project Development Plan}

%\subsection{Development Budget and Funding Sources}

%\subsection{Development Schedule}


\section{Construction Project Definition}

The project includes a few sub-projects, when combined together serve to fulfill the overall project goals.  These include building a facility for the procurement of the low-radioactivity underground argon, a facility for the purification and isotopic separation of argon and other elements, and two detectors that will search for \WIMP\ dark matter between two different mass ranges.  The full work breakdown structure (WBS), including all of the sub-projects, has been developed and agreed upon by the parties responsible for the items, which combined covers the development, procurement, installation and commissioning of the various facilities.  

Included with the preparation of the WBS, was the preparation of a resources-loaded schedule focusing on the successful completion of milestones and deliverables.  This takes into account the funding profile for awards already secured and those which are currently being sought.  The basis of estimates that have been used to establish the baseline budget have been supplied in each case where necessary and have been vetted by the internal project management.  

The full WBS, schedule and funding profile has been reviewed and vetted by the INFN CTS and INFN CSN2, which are the two committees operating under joint charge from the NSF.  For the full proposal Project Execution Plan, the entire WBS and resource-loaded schedule will be elaborated on, focusing on the contribution that this request brings to the project. 

%\subsection{Summary of Total Project Definition}

%\subsection{Work Breakdown Structure (WBS)}

%\subsection{WBS Dictionary}

%\subsection{Scope Management Plan and Scope Contingency}

%\subsection{Cost Estimating Plan, Cost Reports, and Baseline Budget}

%\subsection{Budget Contingency}

%\subsection{Cost Book, Cost Model Data Set, and Basis of Estimate}

%\subsection{Funding Profile}

%\subsection{Baseline Schedule Estimating Plan and Integrated Schedule}

%\subsection{Schedule Contingency}


\section{Staffing}

The resource loaded schedule that has been prepared takes into account the manpower required for the successful completion of each of the tasks.  While much of the manpower already exists within the research groups and is supported through NSF and other agency grant awards, there is a small amount of staffing that will be required to fill some technician positions, as well as training of new students and postdocs to perform project related-tasks.  Based on previous experience within the project, the same methods for staffing the necessary positions will be followed as in the past.  

%\subsection{Staffing Plan}

%\subsection{Hiring and Staff Transition Plan}


\section{Risk and Opportunity Management}

The project has developed two specific risk assessments of the major hazards for the project, a preliminary risk assessment and a quantitative risk assessment.  The preliminary risk assessment consists of an analysis performed applying qualitative and semi-quantitative methods and techniques such as hazards identification (HazID), hazard and operability study (HazOp) and failure modes and effects criticality analysis (FMEA/FMECA).  Using these techniques, major events and risks have been identified and defined, and a first risk characterization has been developed.  The definition and adoption of preventative and protective measures has been developed.

The quantitative risk analysis refines, upgrades and deepens the PRA to a more detailed level.  The QRA consists of an analysis performed applying quantitative methodologies and techniques such as: fault tree analysis (FTA), event tree analysis (ETA) and consequence modeling analysis (CMA).  The main potential adverse events, already identified, will be quantified with a proper likelihood of occurrence.  The identification and determination of the ``event occurrence number'' allows the evaluation of the acceptability of risks, with reference to the risks matrix.

%\subsection{Risk Management Plan)

%\subsection{Risk Register}

%\subsection{Contingency Management Plan}


\section{Systems Engineering}

Systems engineering responsibility falls within each of the project working groups, and is largely at a very advanced stage for each of the systems.  The working groups are each responsible for the fully engineered design of a system, taking into account all project requirements, interfaces, and feasibility for deployment and operations.  Designs that have been finalized and approved by the technical board are then subjected to an external committee review, consisting of a few members from the project team, plus engineers and scientists from national labs that are qualified to assess such systems.  It is the job of the project Technical Coordinator to ensure that systems engineering is done in collaboration across working groups when necessary, while each of the working group leaders is responsible for delivering the design.  The Technical Coordinator will ensure that the working group leaders are aware of all interfaces and have clear expectations of how to interface their system with others.  

Plans for QA/QC are put in place during the engineering of each system, and each is described in the WBS by its own line item.  As the systems are developed and then fabricated, the required tests and checks will be made to ensure that specifications are always achieved per the responsibilities in the WBS.

%\subsection{Systems Engineering Plan}

%\subsection{Systems Engineering Requirements}

%\subsection{Interface Management Plan}

%\subsection{QA/QC Plans}

%\subsection{Concept of Operations Plan}

%\subsection{Facility Divestment Plan}


\section{Configuration Control}

The Technical Coordinator is responsible for the configuration control plan and the process for managing changes to said plan.  The technical coordinator will keep track of project related documentation, ensure appropriate approvals received, and notify the collaborators when changes and updates have been made.  The changes themselves will be vetted by the technical board before implementation into the official documents.  

%\subsection{Configuration Control Plan}

%\subsection{Charge Control Plan}

%\subsection{Document Control Plan}


\section{Acquisitions}

The acquisition process of each of the major pieces of equipment is well documented, and in some cases the process has even already begun.  The schedule and approval process for each of the acquisitions is well understood and will be elaborated on in the full PEP.

%\subsection{Acquisitions Plans}

%\subsection{Acquisition Approval Process}


\section{Project Manangement Controls}
%{\it Note: the scope, complexity, budget profile, and duration of a project should be assessed to determine the need for Earned Value reporting}

The project management structure has already been defined, and is structured with well-defined responsibilities for each of the governance groups and clear roles for each of the key personnel involved.  

The international collaboration supporting the DarkSide project has a well-established project management scheme which includes a top-down flow of information, stemming from the executive (institutional) board.  This board consists of one representative from each of the participating institutions and is responsible for the governing of the project.  Outside of this board, the project contains working groups and committees which report back the progress of tasks related to the project to this governing institutional board, where critical decisions can then be made.  Management structure related decisions are made within the institutional board.  

%\subsection{Project Management Control Plan}

%\subsection{Earned Value Management System (EVMS) Plan}

%\subsection{Financial Business Controls}


\section{Site and Environment}

The various sites for the project have been selected based on the requirements laid out by the project objectives, and the ability for each of the sites to provide the resources for successful implementation of the project.  All locations have now been decided and agreements for siting equipment and system have been established.  Further more, the permitting process for each of the locations has begun and will proceed as called out in the project schedule.  

%\subsection{Site Selection}

%\subsection{Environmental Aspects}


\section{Cyberinfrastructure}
%{\it Note: Proposed Mid-scale RI-1 projects that are focused on Cyberinfrastructure should use Secitons B and C to fully describe the project.  In all cases, proposals that include the use of existing external shared cyberinfrastructure including computing, data, software and networking infrastructure and resources should discuss that utilization here}

The security of data, hardware, and networks will be performed by the institutions and facilities that are involved with the project.  Support will be given to the collaboration by each of the institutional IT departments to ensure the safe storage and processing of all data and related information.  Software will be developed using repository-based systems which will track versions and allow for wide distribution of the code.  The Computing working group is responsible to track the development of the software and inform the collaboration of any major changes and upgrades.  They are also responsible for the implementation of the data management plan for the long term storage and safe keeping of the raw data and produced products.  


%\subsection{Cyber-Security Plan}

%\subsection{Code Development Plan}

%\subsection{Data Management Plan}


\section{Environmental, Safety and Health}

The Collaboration aims at establishing outstanding performance in matters of Health, Safety and Environment (HSE).  The whole experimental collaboration is involved in the effort to achieve these targets.  The collaboration will rely on skilled researchers, technicians and professionals with proven experience in the design, construction, commissioning and operation of complex plants.  All aspects of the experiment, both technical and managerial, are handled in conformity to all applicable US, European and Italian HSE Regulations and Standards, as well as the national laboratory internal guidelines and procedures. 

The collaboration continues to develop and adopt its own policy statements and safety management planning documents, in accordance with the ISO and OHSAS code.  In cooperation with the laboratory and site HSE Staff, Environmental and Safety handbooks, procedures and operating instructions are developed and carefully applied for relevant activities and for the management and operation of plants.  A logbook will be continuously updated with the daily operations performed.  Furthermore, information, education, and training of all Collaboration members are crucial aspects.

Safety management and related information and procedures described are specific to the design, approval, installation, and commissioning of the various components which make up the entire project.  A safety risk assessment will be performed and fully analyzed before the construction and installation phases begin.  The risk analysis methodologies used will be both qualitative and quantitative.  This will be carried out by the Collaboration, under the guidance of the laboratory safety personnel.  Contributions to the safety risk assessment will be made by expert personnel from the funding agencies, member Institutions of the collaboration, and also external consultants with proven expertise in the field of risk assessment and analysis.  This tight cooperation ensured successful results in the past activity of \DSf; the aim is to follow the same path while striving to enhance, if possible, the performance in all  matters.

%\subsection{Environmental, Safety and Health Plans}


\section{Review and Reporting}

Project reporting will occur as called out by each of the partnering funding agencies, and will include notifications for specific events and periodic reports on progress and project technical and financial status.  The necessary reviews, audits, and assessments of the project and its progress are well understood and will be detailed in the full PEP.

%\subsection{Reporting Requirements}

%\subsection{Audits and Reviews}


\section{Integration and Commissioning}
%{\it Note: If the project will be integrated into a larger facility or instrument, the proposal should include a section discussing planned system engineering activities}

The plans for the integration of the various sub-systems into the different facilities involved in the project are well developed.  Each of the different project sites have their own integration plan and methodology for commissioning the systems there.  The criteria for approving the readiness of each of the system for operations has been defined within each of the working groups, and will vetted and approved by the technical board and required external committees prior to the construction of each system.  Systems being delivered by vendors have clear definition of the specifications that must be met and the procedures for acceptance of the system by the project team.

%\subsection{Integrations and Commissioning Plan}

%\subsection{Acceptance/Operational Readiness Plan}


\section{Project Close-out}

Plans for closing out the implementation part of the project and moving to operations are firmly in place and understood within the collaboration.  Support for the operations portion of the project has already been awarded by the funding agencies through other grant awards, NSF as well as international partners.  

%\subsection{Project Close-out Plan}

%\subsection{Transition to Operations Plan}



