%---
\section{Foreign Collaborator Contribution}
\label{sec:foreigncollaborators}
%It is a joint force of the \ArDM, \DSf, \DEAP, and \mCLEAN\ collaborations with the participation of three national underground laboratories (\LNGS, \SNOLAB, and Canfranc) and a partnership with the neutrino platform program at \CERN. 

%XXXXX most notably with the siting of \DSfs\ at the Gran Sasso National laboratory in Italy and the significant participation and funding contribution from the INFN. The increasing role of INFN in DarkSide activities led to the submission of a joint \DSks\ proposal to INFN and the NSF. In addition to the strong collaboration with Italy, early \DS\ collaborators included groups from China, Poland, Russia, and Ukraine, and eventually expanded to groups from France, Brazil, Canada, Greece, Mexico, Germany, Romania, Spain, Switzerland,  and the UK. 
%This expansion coincided with the formation of the \GADMC\, during which DarkSide joined forces with the miniCLEAN program at SNOLAB, Canada, and the ArDM program at Canfranc, Spain. 
%The GADMC as it exists today was cemented with the merging of the DEAP program in 2017, with the participation of three national underground laboratories (Gran Sasso, SNOLAB, and Canfranc), and a partnership with the LAr neutrino program at CERN. 

The \GADMC\  has been characterized by a strong international contribution since its inception in 2017. The current collaboration counts 371 scientists and engineers from 59 institutions and laboratories from 15 countries as members. 
The responsibilities of foreign institutions include:
\begin{itemize}
\item{Italy (\INFN)}: \DSks\ detector (laboratory logistics, argon veto, photoelectronics, \DAQ, computing, mechanical integration, simulations, material screening), \ReD, \Aria, \Urania\ (plant design and delivery), and \DArT.
\item{Italy (Regione Sardegna)}: \Aria.
\item{Italy (Regione Abruzzo and Italian Government)}: \NOA.
\item{Canada}: \DSks\ detector (acrylic vessel, \TPB\ coating, conductive film development, electronics and \DAQ) and \Urania\ (underground argon transportation and storage).
\item{Poland}: \DSks\ detector (material screening and radon control).
\item{Spain}: \DSks\ detector (TPC fabrication, material screening) and \DArT.
\item{France}: \DSks\ detector (simulations, veto), \ARIS, and \ReD.
\item{Russia}: \DSks\ detector (background simulations, veto development, ultra low radioactive  material development and screening, calibration systems development, development of machine learning algorithms for particle identifications, Monte-Carlo and offline software).
\item{China}: \DSks\ detector (acrylic procurement).
\item{U.K.}: \DSks\ detector (veto, photoelectronics).
\end{itemize}

Further information can be found in the Project Execution Plan.