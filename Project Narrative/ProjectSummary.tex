%---
\label{sec:ProjectSummary}

\begin{center}
{\bf Mid-scale RI-1 (M1:IP): DarkSide} \\
(Directorate for Mathematical and Physical Sciences/Division of Physics)
\end{center}
\vspace*{-3.5mm}
{\underline {\bf OVERVIEW:}}
Gravitational effects that cannot be explained by visible matter are well-documented, though their source remains unknown.  A well-motivated explanation for these observations is the existence of an as-yet-undiscovered elementary Weakly Interacting Massive Particle (\WIMP).  The motion of galactic halo \WIMPs\ relative to a detector on Earth could result in \WIMP-nucleus elastic collisions detectable by a low-background, low-threshold detector capable of unambiguously identifying a small number of nuclear recoils over the course of a very large exposure.

This proposal requests support for the \DSk\ (\DSks) detector, a liquid argon time projection chamber (\LArTPC) designed to achieve leading sensitivity to high-mass (above \DSkHighMassThreshold) \WIMP\ dark matter, and \Urania\ plant  for the high volume extraction of low-radioactivity argon from an underground source (\UAr). The \DSk\ project is being pursued by the Global Argon Dark Matter Collaboration (\GADMC), a unification of the \DS, \DEAP, \mCLEAN, and \ArDM\ collaborations into a single, world-wide effort focused on argon dark matter searches. The project was approved in 2017 by the US \NSF\ and the Italian \INFN\ and is officially supported by \LNGS, \LSC, and \SNOLAB.  The \DSks\ Project Execution Plan was approved by an international committee charged by \INFN\ and \NSF\ prior to the experiment's approval and a revised and updated version of the plan is included in this proposal. This proposal requests funding for pieces of infrastructure that are the direct responsibility of the US \NSF-funded groups. 

The \DSks\ experiment builds on the success of the \DSf\ (\DSfs) and \DEAP\ experiments, that have performed background-free searches for WIMP dark matter using large exposures of liquid argon.  \DSfs\ has demonstrated that using \UAr\ lowers the rate of \ce{^39Ar} events by a factor of \DSkDArThreeNineDepletion. \DEAP\ has shown that electron recoil background events in the region of interest can be identified with discrimination better than 1 part in \DEAPPSDRejection. Combined results from the two \DSfs\ runs demonstrate the ability of large \LArTPCs\ to operate in an ``instrumental background-free  mode,'' a mode in which fewer than \BackgroundFreeRequirement\ (other than nuclear recoils from elastic scattering of atmospheric and diffuse supernova background neutrinos) are expected in the region of interest for the planned exposure of \DSks\ detector.

\vskip0.05in
{\underline {\bf INTELLECTUAL MERIT:}}
The \DSks\ detector will have ultra-low background and the ability to measure its background rates {\it in situ}, resulting in an expected sensitivity to \WIMP-nucleon cross sections of \DSkExtendedSensitivityOneGeVUnit\ (\DSkExtendedSensitivityTenGeVUnit) for \ \WIMPMassOneTev\ (\WIMPMassTenTev) \WIMPs\ with a total exposure of \DSkExtendedExposure.  The projected \SI{5}{\sgm} discovery reach extends a factor of \DSkHighMassSensitivityImprovementOverLZ\ below that of \LZ.  \DSks\ will either detect \WIMP\ dark matter or exclude a large fraction of the favored parameter space. It will be sensitive to a galactic supernova neutrino burst. It will also lay the groundwork for a future, multi-hundred tonne argon experiment, \Argo, designed to complete the search for \WIMPs\ through the so-called ``neutrino floor'' and measure low-energy solar and supernova neutrinos with high precision. The infrastructure of \DSk\ project will also enable \DSl, which, building upon the world-leading low-mass dark matter results of \DSfs, is expected to dominate searches for WIMPs with masses below \DSlLowMassThreshold.


\vskip0.05in
{\underline {\bf BROADER IMPACT:}}
Scientific broader impacts of the project include the discovery of a novel, commercially viable helium source that today supplies \UraniaHeNationalReserveFractionEquivalentRate\ of the US production; the production of hundreds of tonnes of low-radioactivity \UAr\ for \DSks\ as well as for other technical uses including nuclear test ban verification and radiometric dating; and the development of low-background, large-area, single-photon, cryogenic photosensors.  The planned \Aria\ project for \UAr\ purification may improve the worldwide availability of valuable stable rare isotopes such as \ce{^18O}, \ce{^15N}, and \ce{^13C}, which are used for various medical, industrial, and energy generation applications.  \LArTPC\ technology has led to \ThreeDPi, an innovative, patent-pending LAr-based TOF-\PET\ system that can enhance cancer screening sensitivity while dramatically lowering patient radiation dose.  

Specific E\&O programs are planned as part of the program with a focus on educating K-12 teachers about basic physics and its relation to dark matter detection, re-starting a summer school experience for high-school and undergraduate students, and giving education and training opportunities to undergraduate students at participating underrepresented-minority serving institutions.